\documentclass[a4paper]{article}

\usepackage{fontspec}
\usepackage{graphicx}
\usepackage{hyperref}
\usepackage{cleveref}
\usepackage{titlesec}
\usepackage{titling}
\usepackage{minted}
\usepackage{natbib}
\usepackage{xcolor}
\usepackage{subcaption}

\usepackage[shorthands=off,main=spanish]{babel}

\renewcommand{\listingscaption}{Programa}
\def\listingautorefname~#1\null{(Programa #1)\null}
%\def\figureautorefname~#1\null{(Figura #1)\null}
\def\figureautorefname~#1\null{(#1)\null}

\newcommand{\sectionbreak}{\clearpage}

\pretitle{\begin{center} \includegraphics[width=200pt]{usal.png} \\ \vspace{3em} \LARGE}
\title{Práctica 2 con Arduino \\ \large Periféricos}
\author{Emilio Cobos Álvarez y Christian Bonal Martín}
\date{11 de Abril de 2016}

\begin{document}

\maketitle

\clearpage
\tableofcontents
\setlength{\parskip}{1em}
\setlength{\parindent}{0pt}
\newpage

\section{Sensor flexor}

\subsection{Sin LEDs}

Lo primero que hicimos fue mirar el
\href{https://cdn.sparkfun.com/datasheets/Sensors/ForceFlex/FLEX\%20SENSOR\%20DATA\%20SHEET\%202014.pdf}{datasheet}
del sensor, y ver la salida que nos daba, sin ningún tipo de LED como salida:

\begin{listing}[h]
\inputminted[frame=single, framesep=.5em, linenos, rulecolor=\color{gray}]{java}{../flex-sensor/flex-sensor-1.ino}
\end{listing}

\begin{figure}
  \includegraphics[width=\linewidth]{../img/flex-basic-circuit.jpg}
  \caption{Circuito flexor básico}
  \label{fig:flex-basic-circuit}
  \centering
  \begin{subfigure}{.4\linewidth}
    \includegraphics[width=.9\linewidth]{../img/flex-middle.png}
    \caption{Salida en reposo}
    \label{fig:flex-middle-out}
  \end{subfigure}
  \begin{subfigure}{.4\linewidth}
    \includegraphics[width=.9\linewidth]{../img/flex-left.png}
    \caption{Salida doblado hacia la izda.}
    \label{fig:flex-left-out}
  \end{subfigure}
  \begin{subfigure}{\linewidth}
    \includegraphics[width=.9\linewidth]{../img/flex-right.png}
    \caption{Salida doblado hacia la dcha.}
    \label{fig:flex-right-out}
  \end{subfigure}
\end{figure}

\subsection{Con LEDs}

La versión con leds del programa lo que hace es encender uno de los LEDs de
salida u otro dependiendo de en qué dirección esté flexionado. El código es
relativamente sencillo, ya que mapeamos los valores entre 0 y 1, y dependiendo
de si es mayor que 0.5 o no encendemos uno de los dos LEDs.

\begin{listing}[h]
\inputminted[frame=single, framesep=.5em, linenos, rulecolor=\color{gray}]{java}{../flex-sensor/flex-sensor.ino}
\end{listing}


\bibliographystyle{plainnat}
\bibliography{bibliography}

\end{document}
