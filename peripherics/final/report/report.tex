\documentclass{article}

\usepackage{fontspec}
\usepackage{gensymb}
\usepackage{graphicx}
\usepackage{hyperref}
\usepackage{cleveref}
\usepackage{titlesec}
\usepackage{titling}
\usepackage{minted}
\usepackage{natbib}
\usepackage{xcolor}
\usepackage{subcaption}
% \usepackage[a4paper, top=1in, bottom=1in]{geometry}

\usepackage[shorthands=off,main=spanish]{babel}

\renewcommand{\listingscaption}{Programa}
\def\listingautorefname~#1\null{(Programa #1)\null}
%\def\figureautorefname~#1\null{(Figura #1)\null}
\def\figureautorefname~#1\null{(#1)\null}

% \newcommand{\sectionbreak}{\clearpage}

% Make paragraphs like subsubsubsections
\setcounter{secnumdepth}{4}
\titleformat{\paragraph}
{\normalfont\normalsize\bfseries}{\theparagraph}{1em}{}
\titlespacing*{\paragraph}
{0pt}{3.25ex plus 1ex minus .2ex}{1.5ex plus .2ex}

% Hack for multipage code
\newenvironment{multipagecode}{}{}

\pretitle{\begin{center} \includegraphics[width=200pt]{usal.png} \\ \vspace{3em} \LARGE}
\title{Práctica final con Arduino \\ \large Periféricos}
\author{Emilio Cobos Álvarez \and Christian Bonal Martín}
\date{9 de Mayo de 2016}

\begin{document}

\maketitle

\clearpage
\tableofcontents
\setlength{\parskip}{1em}
\setlength{\parindent}{0pt}
\newpage

\section{Objetivo y descripción de la práctica}

El objetivo de la práctica es realizar una mini-estación meteorológica para
poder monitorizar la temperatura y la humedad de un lugar.

Para ello usaremos diversos sensores como periféricos de entrada (el TMP-36 para
la temperatura, y el DHT-11 para la humedad), y una pantalla LCD como periférico
de salida.

Como pequeña funcionalidad extra, se ha añadido una advertencia en caso de que
la temperatura se desvíe de un rango de valores determinado.

\section{Descripción del circuito}

Se ha usado como placa una placa \textit{Arduino Uno} como en las prácticas
anteriores.

El circuito es relativamente estándar, con la pantalla LCD conectada entre los
pines digitales 2 y 12, un RTD TMP-36 conectado a la entrada analógica
\texttt{A0}, y el sensor de humedad DHT-11 conectado al pin digital 13.

\begin{figure}[H]
  \centering
  \begin{subfigure}{.4\linewidth}
    \includegraphics[width=\linewidth]{../img/final-circuit.jpg}
    \caption{Diagrama del circuito}
    \label{fig:final-circuit-diagram}
  \end{subfigure}
  \begin{subfigure}{.4\linewidth}
    \includegraphics[width=\linewidth]{../img/circuit.jpg}
    \caption{Circuito en funcionamiento}
    \label{fig:final-circuit}
  \end{subfigure}
\end{figure}

\subsection{Pantalla LCD}

La pantalla LCD utilizada ha sido una pantalla estándar de 16 columnas por 2
filas. Se ha utilizado la biblioteca \texttt{LiquidCrystal} de Arduino para su
manejo.

\subsubsection{Funcionamiento}

Las pantallas LCD funcionan encendiendo o apagando píxels digitalmente, siendo
cada píxel uno de los diferentes puntos que conforman la imagen final mostrada
en la pantalla.

En esta pantalla, los píxels son sólo o blancos o negros (encendidos o
apagados), pero se podrían hacer de color con varios subpíxels por cada píxel
(uno rojo, uno azul, y uno verde).

Las pantallas LCD usan cristal líquido combinado con cristales polarizados.

\paragraph{El cristal líquido}

El cristal líquido, irónicamente, no es líquido como tal, sino una especie de
estado de la materia entre líquido y sólido. La parte curiosa del cristal
líquido es que cuando se le aplica una corriente eléctrica su estructura cambia,
de estar más doblada a más recta.

Esto se usa para ``desviar la luz", o más adecuadamente hablando, cambiar la
dirección de oscilación del campo electromagnético de la luz, como veremos más
adelante.

\paragraph{El cristal polarizado}

Intuitivamente, un cristal polarizado sólo deja pasar la luz que pasa a través
de él perpendicularmente. Esta es una definición muy imprecisa pero suficiente
para entender cómo funcionan las pantallas LCD.

No obstante, a continuación daremos un poco más de fondo a esa definición.

La luz es una onda electromagnética transversal, lo que quiere decir que la
oscilación del campo electromagnético es perpendicular a su dirección de
propagación.

La polarización de una onda es la propiedad de que esa onda pueda oscilar con
más de una orientación.

La mayoría de fuentes de luz emiten luz con campos eléctricos
\footnote{En una onda electromagnética el campo eléctrico y el magnético que la
forman oscilan perpendicularmente entre sí. Cuando hablamos de polarización se
estudia por convención el campo eléctrico, porque el vector de campo magnético
se puede obtener a partir del eléctrico.} en cualquier  dirección de todas las
posibles direcciones perpendiculares a la de propagación.

Llamaremos \textbf{luz polarizada a aquella que tiene una determinada dirección
de oscilación}.

Una de las formas de obtener luz polarizada, es usar un filtro
polarizador\cite{wiki:polarizer}, que
sólo deja pasar la luz que oscila en una sola dirección.

\paragraph{Combinación}

Una pantalla LCD funciona con una luz brillante continuamente alumbrando detrás
de la pantalla, y varios píxels hecho cada uno con \textbf{dos filtros
polarizadores, y una pequeña cantidad de cristal líquido entre ellos}.

Cuando hay corriente a través del cristal líquido, deja pasar la luz
perfectamente, así que \textbf{la luz polarizada que deja pasar el primer filtro
la deja pasar el segundo también, y el píxel parece claro}.

Sin embargo, cuando no hay corriente, el cristal líquido cambia de estructura,
desviando la dirección de la luz, y esto hace que \textbf{el segundo cristal
polarizado la bloquee, apareciendo el píxel oscuro}.

Así, la pantalla LCD es un circuito integrado que mediante una serie de pines
más o menos conveniente que permite controlar la electricidad que se le da a los
transistores de cada píxel.

\subsubsection{Programa de ejemplo}

Se adjunta un programa de ejemplo de la pantalla LCD que imprime el típico
mensaje ``Hello, world!" por la pantalla.

\begin{listing}[H]
\inputminted[frame=single, framesep=.5em, linenos,
  rulecolor=\color{gray}]{cpp}{../vis/vis.cpp}
\caption{Programa de ejemplo para la pantalla LCD}
\end{listing}

\subsection{Sensor de humedad y temperatura DHT-11}

El DHT-11 es un sensor básico de humedad y temperatura de costo reducido. Usa un
sensor de capacidad para medir la humedad y un termistor para medir la
temperatura del aire que lo rodea.

Está diseñado para medir temperaturas entre $0$ y $50\degree C$ con una
precisión de $\pm 2\degree C$ y para medir humedad entre $20\%$ y $80\%$ con una
precisión de $5\%$ con periodos de muestreo de 1 segundo.

Sólo lee cantidades enteras, luego no podremos leer una temperatura o una
humedad con decimales.

\subsubsection{Funcionamiento}

El DHT-11 requiere su propio protocolo para comunicarse a través de un sólo
hilo. Este protocolo es simple y puede implementarse usando los pines de I/O en
cualquier microcontrolador.

El microcontrolador debe iniciar la comunicación con el DHT-11 manteniendo la
linea de datos en estado bajo durante al menos $18ms$. Luego el DHT-11 envía una
respuesta con un pulso a nivel bajo de $80\mu s$ y luego deja ``flotar" la línea
de datos por otros $80\mu s$.

Los datos binarios se codifican según la longitud del pulso alto. Todos los bits
comienzan con un pulso bajo de $50\mu s$.

Luego viene un pulso alto que varía según el estado lógico o el valor del bit
que el DHT-11 desea transmitir.

Se utilizan pulsos de $26$ a $28\mu s$ para un $0$ y pulsos de $70\mu s$ para un
$1$. Los pulsos se repiten hasta un total de 40 bits.

En cuanto a los datos, se transmiten 40 bits (5 bytes) en total, cuyo valor se
describe a continuación:

\begin{itemize}
  \item La parte entera de la humedad relativa (\textit{RH}).
  \item La parte decimal de la humedad relativa (no se utiliza, siempre es 0).
  \item La parte entera de la temperatura.
  \item La parte decimal de la temperatura (no se utiliza, siempre es 0).
  \item El último byte es la suma de comprobación (checksum), resultante de
    sumar todos los bytes anteriores.
\end{itemize}

\subsubsection{Programa de ejemplo}

Usamos la librería GPLv3 encontrada en la
\href{http://playground.arduino.cc/main/DHT11Lib}{página oficial de Arduino}. Se
incluye en la entrega de la práctica, y se puede comprobar cómo el código se
corresponde con el protocolo descrito anteriormente, con algunas diferencias
mínimas:

\begin{itemize}
  \item Se esperan $5ms$ como tiempo de espera, citando al autor: ``to be safe".
  \item El límite para que la respuesta sea considerada un $1$ es exactamente
    $40ms$ (que está dentro de lo esperado dada la especificación del
    protocolo).
\end{itemize}

\begin{listing}[H]
\inputminted[frame=single, framesep=.5em, linenos,
  rulecolor=\color{gray}]{cpp}{../dht11/dht11.cpp}
\caption{Programa de ejemplo para el DHT-11}
\end{listing}

\subsection{Sensor de temperatura TMP-36}

Ya usamos este sensor en la primera sesión de prácticas, así que sirva esto como
recapitulación.

\subsubsection{Funcionamiento}

El TMP-36 es un RTD o \textit{Resistance Temperature Detector}
(\cite{wiki:rtd}).

Algunos de los RTD pueden usar termistores, pero a diferencia de estos, la
relación que tiene el voltaje de salida ($V_{out}$) con la temperatura es
lineal, en vez de exponencial. Esto no sólo simplifica el tratamiento de sus
datos, y también suelen tener mucho más rango de temperaturas que los
termistores.

Si se quiere más rango aún de temperaturas, otra opción podría ser un
\href{https://es.wikipedia.org/wiki/Termopar}{termopar}.

La fórmula que relaciona la temperatura en este modelo, según
\href{http://www.analog.com/media/en/technical-documentation/data-sheets/TMP35_36_37.pdf}
{la hoja de datos del sensor} (\cite{web:tmp36-data-sheet}) es:

$$T = \frac{V_{out} - 500}{10}$$

Donde $T$ está en grados centígrados, y $V_{out}$ en milivoltios.

\subsubsection{Programa de ejemplo}

\begin{listing}[H]
\inputminted[frame=single, framesep=.5em, linenos,
  rulecolor=\color{gray}]{cpp}{../../arduino-1/temperature/temperature.ino}
\caption{Programa de ejemplo para el RTD}
\end{listing}

\clearpage
\section{Programación de la placa}

Teniendo todo lo anterior, realizar el programa final es cuestión de acoplar
los programas anteriores.

Como técnica para evitar variaciones drásticas de temperatura reportadas por el
sensor TMP-36, se usa una caché estática con las cuatro últimas lecturas, y se
hace la media de las mismas, en la función \texttt{Tmp36AvgTemperature}.

Además de eso, se ha añadido la funcionalidad para alertar si la temperatura se
sale de un rango determinado (ver \texttt{TEMPERATURE\_MAX} y
\texttt{TEMPERATURE\_MIN}), en cuyo caso se mostrará una exclamación parpadeando
en la parte derecha de la pantalla.

\begin{multipagecode}
\inputminted[frame=single, framesep=.5em, linenos,
  rulecolor=\color{gray}]{cpp}{../final/final.cpp}
\captionof{listing}{Programa final}
\end{multipagecode}

\section{Dificultades y observaciones}

\subsection{¿Por qué se ha usado C++ en vez de la IDE de Arduino?}

Arduino tiene un IDE que no está demasiado mal, pero el hecho de que en Linux no
funcione tan bien como debería, y que editar los archivos desde un editor
externo conlleva una serie de pasos bastante larga son razones grandes en su
contra.

En particular, \textbf{añadir una librería no funcionaba correctamente en
nuestro sistema}.

Esto fue razón más que de sobra para que, aprovechando el ejemplo del código de
la librería que encontramos para el DHT-11, hiciéramos el cambio a C++.

Como ventaja extra, hemos aprendido una gran cantidad de cosas acerca del
funcionamiento de la plataforma Arduino y de todas las utilidades de consola que
proporciona.

\section{Otros adjuntos}

Aparte del código como es habitual, y todo el contenido del informe, se adjunta
un vídeo del circuito funcionando (en la carpeta \texttt{video}).

\bibliographystyle{plainnat}
\bibliography{bibliography}

\end{document}
