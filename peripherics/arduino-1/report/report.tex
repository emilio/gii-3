\documentclass[a4paper]{article}

\usepackage{fontspec}
\usepackage{graphicx}
\usepackage{hyperref}
\usepackage{cleveref}
\usepackage{titlesec}
\usepackage{titling}
\usepackage{minted}
\usepackage{natbib}
\usepackage{xcolor}

\usepackage[shorthands=off,main=spanish]{babel}

\def\listingautorefname~#1\null{(Listing #1)\null}
%\def\figureautorefname~#1\null{(Figura #1)\null}
\def\figureautorefname~#1\null{(#1)\null}

\newcommand{\sectionbreak}{\clearpage}

\pretitle{\begin{center} \includegraphics[width=200pt]{usal.png} \\ \vspace{3em} \LARGE}
\title{Práctica 1 con Arduino \\ \large Periféricos}
\author{Emilio Cobos Álvarez, Christian Bonal Martín y Alejandro Bodego Tomé}
\date{14 de marzo de 2016}


\begin{document}

\maketitle

\clearpage
\tableofcontents
\setlength{\parskip}{1em}
\setlength{\parindent}{0pt}
\newpage

\section{Led parpadeando digitalmente}

Éste es el ejercicio más simple de la programación con Arduino, y como tal no
nos llevó demasiado tiempo. El circuito es directo de montar, y el código es
también simple:

\begin{listing}[h]
\begin{minted}[frame=single, framesep=.5em, linenos, rulecolor=\color{gray}]{java}
void setup() {
  pinMode(13, OUTPUT);
}

void loop() {
  digitalWrite(13, HIGH);
  delay(1000);

  digitalWrite(13, LOW);
  delay(1000);
}
\end{minted}
\end{listing}

Se obvian los detalles de las fotos mostrando que el circuito funciona, porque
se considera trivial.

\section{Circuito detector de luz y oscuridad}

\subsection{Descripción del trabajo realizado}

Este circuito fue mucho más interesante y el principal de la práctica.
Aprendimos a usar la entrada y salida analógica.

\subsubsection{Montaje del circuito}

Nuestro montaje tenía el aspecto de la figura \autoref{fig:light-off} cuando había
luz, dado que la resistencia es menor, y la intensidad que nos llega a la
entrada analógica \texttt{A0} es menor.

\begin{figure}[H]
  \includegraphics[width=\linewidth]{../img/light-dark-off.jpg}
  \caption{Circuito detector de luz funcionando con luz (led apagado).}
  \label{fig:light-off}
\end{figure}

Se puede comprobar en la figura \autoref{fig:light-on} que cuando hay sombra, y
por lo tanto la resistencia fotosensible es mayor, nos llega un valor mayor en
la
entrada \texttt{A0}.

\begin{figure}[H]
  \includegraphics[width=\linewidth]{../img/light-dark-on.jpg}
  \caption{Circuito detector de luz funcionando sin luz (led encendido).}
  \label{fig:light-on}
\end{figure}

\subsubsection{Programación de la placa}

La programación de la placa fue bastante directa. Nos fijamos en varios ejemplos
de la documentación acerca de entrada y salida analógica
(\cite{web:arduino-doc-analog-io}, \cite{web:arduino-doc-pwm}), y lo completamos
\autoref{light-detector-code} sin mayor complicación.

Como se indica en los comentarios, experimentalmente sólo éramos capaces de
obtener valores de entre $350$ y $600$, por eso hubo que hacer ese condicional
para dejar la salida a $0$ si el valor era menor o igual que $400$.

\begin{listing}[H]
\begin{minted}[frame=single, framesep=.5em, linenos, rulecolor=\color{gray}]{java}
// Light-detector circuit and conversion between
// analog and digital signals
//
// Reference:  https://www.arduino.cc/en/Tutorial/PWM

// The input pin where the current passing by
// the ressistance is going to enter
const int ANALOG_INPUT = A0;

// The output pin
const int ANALOG_OUTPUT = 9;

void setup() {
  pinMode(ANALOG_OUTPUT, OUTPUT);
  Serial.begin(9600);
}

void loop() {
  int input_value = analogRead(ANALOG_INPUT);
  int digital = 0;
  // Experimentally we only received values between 300 and ~600.
  // If not this will be the correct code:
  // int digital = map(input_value, 0, 1023, 0, 255);
  if (input_value > 400)
    digital = map(input_value, 300, 650, 0, 255);

  Serial.print("in: ");
  Serial.println(input_value);

  analogWrite(ANALOG_OUTPUT, digital);
}
\end{minted}
\caption{Código usado para el circuito}
\label{light-detector-code}
\end{listing}

\subsection{Funcionamiento del circuito}

El circuito se basa en el funcionamiento del fotorresistor o LDR
(\cite{web:wikipedia-ldr}), una componente cuya resistencia disminuye según
aumenta el nivel de luz sobre ella, ya que al ser los fotones absorbidos por el
semiconductor, los electrones adquieren la energía suficiente para saltar a la
banda de conducción.

Gracias a esta resistencia se consigue variar la caída de voltaje de cada parte
del circuito dependiendo de su valor.

\subsection{Balance de costes}

El hecho de tener una placa programable hace que el montaje del circuito sea
mucho más simple que el montaje previo que hicimos en el laboratorio, que
requería, entre otras cosas, varios amplificadores operacionales
retroalimentados y resistencias.

Este montaje sin embargo sólo necesitaba dos resistencias (aparte del LDR).

No sobra decir que el coste de una placa de estas características es mucho menor
que el de un laboratorio de electrónica, y que además pueden ser adquiridas por
cualquier persona, facilitando a los alumnos el proceso de investigar por su
cuenta.

Por lo tanto, parece que este tipo de placas son, cuanto menos, un alternativa
razonable a los experimentos en el laboratorio.

\section{Referencias comentadas}

Se han usado una buena cantidad de referencias. Aunque enumeradas abajo, se
comentan aquí ya que estaba solicitado explícitamente.

\paragraph{Entrada analógica --- Arduino}

La documentación acerca de la entrada analógica en Arduino
(\cite{web:arduino-doc-analog-io}) ha sido de mucha utilidad para construir el
código de ejemplo, ya que explica con detalle el significado del valor de
retorno de \texttt{analogRead}, y sirve de directorio a otra gran cantidad de
documentación.

\paragraph{Pulse Width Modulation --- Arduino}

La documentación de arduino acerca de PWD (\cite{web:arduino-doc-pwm}) nos ha
servido para entender qué hacíamos realmente al usar la función
\texttt{analogWrite}, que no es otra cosa que modular la amplitud del pulso que
se manda por ese pin.

\paragraph{Ejemplos de Arduino}

Los ejemplos de Arduino que vienen con el IDE han servido de ayuda a la hora de
encontrar determinados patrones, como el uso de la función \texttt{map}.

\section{Ficheros adjuntos}

Todo el código relevante, y las fotos tomadas, están en este informe. De todos
modos, dado lo complicado que se puede hacer copiar código desde un PDF, y la
menor resolución a la que están las imágenes, se adjuntarán junto con este
informe también.

\bibliographystyle{plainnat}
\bibliography{bibliography}

\end{document}
