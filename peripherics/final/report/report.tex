\documentclass{article}

\usepackage{fontspec}
\usepackage{graphicx}
\usepackage{hyperref}
\usepackage{cleveref}
\usepackage{titlesec}
\usepackage{titling}
\usepackage{minted}
\usepackage{natbib}
\usepackage{xcolor}
\usepackage{subcaption}
\usepackage[a4paper, top=1in, bottom=1in]{geometry}

\usepackage[shorthands=off,main=spanish]{babel}

\renewcommand{\listingscaption}{Programa}
\def\listingautorefname~#1\null{(Programa #1)\null}
%\def\figureautorefname~#1\null{(Figura #1)\null}
\def\figureautorefname~#1\null{(#1)\null}

% \newcommand{\sectionbreak}{\clearpage}

\pretitle{\begin{center} \includegraphics[width=200pt]{usal.png} \\ \vspace{3em} \LARGE}
\title{Práctica final con Arduino \\ \large Periféricos}
\author{Emilio Cobos Álvarez, Christian Bonal Martín}
\date{9 de Mayo de 2016}

\begin{document}

\maketitle

\clearpage
\tableofcontents
\setlength{\parskip}{1em}
\setlength{\parindent}{0pt}
\newpage

\section{Objetivo y descripción de la práctica}

El objetivo de la práctica es realizar una mini-estación meteorológica para
poder monitorizar la temperatura y la humedad de un lugar.

Para ello usaremos diversos sensores como periféricos de entrada (el TMP-36 para
la temperatura, y el DHT-11 para la humedad), y una pantalla LCD como periférico
de salida.

Como pequeña funcionalidad extra, se ha añadido una advertencia en caso de que
la temperatura se desvíe de un rango de valores determinado.

\section{Descripción del circuito}

Se ha usado como placa una placa \textit{Arduino Uno} como en las prácticas
anteriores.

El circuito es relativamente estándar, con la pantalla LCD conectada entre los
pines digitales 2 y 12, un RTD TMP-36 conectado a la entrada analógica
\texttt{A0}, y el sensor de humedad DHT-11 conectado al pin digital 13.

% TODO: Imagen del circuito

\subsection{Pantalla LCD}

La pantalla LCD utilizada ha sido una pantalla estándar de 16x2 columnas.

\paragraph{Funcionamiento}

\paragraph{Programa  de ejemplo}


\subsection{Sensor de humedad y temperatura DHT-11}
\subsection{Sensor de temperatura TMP-36}

\section{Programación de la placa}

% TODO: meter en "Conclusión"?
\section{Dificultades y observaciones}

\bibliographystyle{plainnat}
\bibliography{bibliography}

\end{document}
