\section{Internal and external representation of a system}

For convenience we'll call \(u(t)\) to the input and, \(y(t)\) to the output.

Remember that \(\dot{\vec{x}}\) means the derivative of \(\vec{x}\).

\subsection{External representation}
  Just a transference function that associates the input with the output.

\subsection{Internal representation}
  Now we've got two equations: a \textit{state equation} and a \textit{output
  equation}.

  State equation:

  \[ \dot{\vec{x}} = f(u, \vec{x}, t) \]

  Output equation:

  \[ y = g(u, \vec{x}, t) \]

  \subsubsection{Simplification in some kind of systems}

    \[ \dot{\vec{x}} = A * \vec{x} + B * u \]
    \[ y = C * \vec{x} + D * u \]

\section{Linear systems}
  \subsection{Laplace transform}

  \[ f(t) = \int^{\infty}_0 e^{-st} f(t) dt \]

  We calculate the transform of the input \(u(t)\) giving \(U(s)\),
  and the transform of the output \(y(t)\) giving \(Y(t)\).

  We use the property that:

  \[ \frac{f(t)}{\partial t} \implies s \dot F(s) - f(0) \]

  To reach a polynomial relation in the domain of \(s\).

  The laplace transform is linear, so we can apply to the relation in the time
  domain the laplace transform in the \(s\) domain.

    \subsubsection{Example with the dork with mass}


