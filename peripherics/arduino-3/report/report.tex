\documentclass{article}

\usepackage{fontspec}
\usepackage{graphicx}
\usepackage{hyperref}
\usepackage{cleveref}
\usepackage{titlesec}
\usepackage{titling}
\usepackage{minted}
\usepackage{natbib}
\usepackage{xcolor}
\usepackage{subcaption}
\usepackage[a4paper, top=1in, bottom=1in]{geometry}

\usepackage[shorthands=off,main=spanish]{babel}

\renewcommand{\listingscaption}{Programa}
\def\listingautorefname~#1\null{(Programa #1)\null}
%\def\figureautorefname~#1\null{(Figura #1)\null}
\def\figureautorefname~#1\null{(#1)\null}

% \newcommand{\sectionbreak}{\clearpage}

\pretitle{\begin{center} \includegraphics[width=200pt]{usal.png} \\ \vspace{3em} \LARGE}
\title{Práctica 3 con Arduino \\ \large Periféricos}
\author{Emilio Cobos Álvarez, Christian Bonal Martín, Alejandro Bodego}
\date{18 de Abril de 2016}

\begin{document}

\maketitle

\clearpage
\tableofcontents
\setlength{\parskip}{1em}
\setlength{\parindent}{0pt}
\newpage

\section{Servomotor}

En esta práctica de actuadores no pudimos realizar pruebas con el motor DC, pero
lo que sí pudimos hacer es usar el servomotor.

El servomotor es relativamente sencillo, y no digamos ya de utilizar gracias a
las bibliotecas de Arduino.

El ejercicio consistía en lo siguiente:

\begin{itemize}
  \item{Se controlará el funcionamiento de un servo como elemento de salida de
      la siguiente manera: se simulará la apertura de una puerta (giro entre 0 y
      90grados) de tal forma que la puerta se va abriendo a medida que el LDR
      detecta más oscuridad.}
\end{itemize}

\subsection{Montaje del circuito}

Lo primero que hicimos fue montar el circuito esquematizado en el guión de la
práctica para poder experimentar y calibrar el servomotor.

Una vez lo tuvimos montado y calibrado, comprobamos (inicialmente
experimentalmente, y luego mirando
\href{https://www.sparkfun.com/products/9065}{la página del
producto}\footnote{Sí, ya sabemos que el orden debería de haber sido el
inverso, pero...}) que sólo soportaba rotación hasta $160\deg$.

Eso nos llevó a hacer un programa de prueba de tal manera que lo único que hacía
era rotar de $0$ a $160\deg$ y viceversa.

El circuito resultante de nuestro primer intento es el montaje básico:

\begin{figure}[H]
  \includegraphics[width=\linewidth]{../img/servo-basic.jpg}
  \caption{Circuito servo básico}
  \label{fig:servo-basic}
\end{figure}

Posteriormente, ya con el LDR integrado (y con las resistencias apropiadas,
basándonos en las primeras prácticas), el resultado es el siguiente:

\begin{figure}[H]
  \includegraphics[width=\linewidth]{../img/servo-ldr.jpg}
  \caption{Circuito servo con LDR}
  \label{fig:servo-ldr}
\end{figure}

\subsection{Programación de la placa}

\subsubsection{Circuito básico}

La versión básica del programa nos sirvió para comprobar cómo funcionaba el
servo, y no va mucho más allá de probar cómo funcionaba la biblioteca que
proporcionaba Arduino:

\begin{listing}[H]
\inputminted[frame=single, framesep=.5em, linenos,
rulecolor=\color{gray}]{cpp}{../servo/servo-roundtrip.ino}
\caption{Programa básico}
\end{listing}

\subsubsection{Col el LDR}

La versión con el LDR tiene algo más de complejidad, ya que no tomamos sólo los
valores máximos y mínimos que obteníamos en esa habitación, sino que se adapta
con el tiempo, cogiendo un valor en la primera lectura, y hayando el máximo y el
mínimo a partir de ahí, pasando de $0$ a $160\deg$ entre ellos.

\begin{listing}[H]
\inputminted[frame=single, framesep=.5em, linenos,
rulecolor=\color{gray}]{cpp}{../servo/servo.ino}
\caption{Código del programa con el LDR}
\end{listing}

\section{Pantalla LCD}


\section{Referencias comentadas}

La única referencia que hemos utilizado para este informe, aparte de las ya
referenciadas datasheets, es el tutorial de arduino para el servomotor
(\cite{web:arduino-servo}).

No hay tampoco demasiado que comentar, fue muy útil para ponerlo a funcionar
rápido.

\section{Otros adjuntos}

Aparte del código como es habitual, y todo el contenido del informe, se adjunta
un vídeo del funcionamiento del servomotor, en la carpeta \texttt{video}.

\bibliographystyle{plainnat}
\bibliography{bibliography}


\end{document}
