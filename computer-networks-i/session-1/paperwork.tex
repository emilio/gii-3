\documentclass{article}

\usepackage{fontspec}
\usepackage{titlesec}
\usepackage{enumitem}
\usepackage{tabularx}
\usepackage{minted}
\usepackage[spanish]{babel}
\usepackage[cm]{fullpage}

\usemintedstyle{friendly}

\setlength{\parindent}{2em}
\setlength{\parskip}{.5em}

\titleformat{\section}[hang]{\normalfont\bfseries\Large}{Bloque \thesection: }{0pt}{}
\titleformat{\subsubsection}{\normalfont\it}{\thesubsubsection) }{10pt}{}
\titleformat{\paragraph}[hang]{\normalfont\bfseries}{\thesubsubsection) }{0}{}
\titlespacing{\paragraph}{1em}{0pt}{0pt}

\renewcommand{\thesection}{\Alph{section}}
\renewcommand{\thesubsubsection}{\alph{subsubsection}}
\renewcommand{\thesubsection}{\arabic{subsection}}

\setlist[description]{leftmargin=\dimexpr2.5\parindent,itemindent=\dimexpr-1\parindent}

\title{Práctica 1: El puerto serie}
\author{Emilio Cobos Álvarez (B1) \\ Anna Elena Chesnais (B1) \\
        Cristina García González (B1)}

\begin{document}
\maketitle

\newpage
\section{El cable de módem nulo}
\subsection{Construya un cable de módem nulo con el fin de comunicar dos
            equipos por el puerto serie (recomendada versión completa).}
\subsubsection{Use el programa \emph{PuTTY} para comprobar dicha comunicación.
               ¿Qué parámetros se pueden modificar en el puerto serie?.
               ¿Qué valores son posibles en cada uno de ellos?.
               Pruebe distintas configuraciones (al menos dos) y describa lo
               que ocurre al teclear caracteres en una y otra consola.
               ¿Se puede seleccionar el conjunto de caracteres que el terminal
               remoto está utilizando?. ¿Para qué sirve esto?.}
  \paragraph{Parámetros modificables y valores aceptados}
    \begin{description}
        \item[Puerto:]
          Cualquier nombre de archivo que represente un puerto de serie.
        \item[Pulsos por segundo (baudios):]
          Cualquiera de los valores estándares que enumeramos a continuación: \\
            110,
            300\footnote{Valor mínimo con el que pudimos transmitir datos},
            600, 1200, 2400, 4800, 9600, 14400, 19200, 28800, 38400,
            56000, 57600, 115200, 128000,
            153600 \footnote{El valor máximo que nuestra conexión soportaba},
            230400, 256000, 460800 y 921600.
        \item[Bits de datos de un carácter:]
          5, 6, 7 u 8 bits.
        \item[Bits de stop:]
          1, 1.5 ó 2.
        \item[Paridad:]
          Ninguna, par o impar.
        \item[Control de flujo:]
          Ninguno, \texttt{XON/XOFF} o \texttt{RTS/CTS}.
      \end{description}
  \paragraph{Configuraciones probadas}
    \begin{description}
      \item[Configuración por defecto:] \hfill \\
        9600 baudios, sin paridad, control de flujo \texttt{XON/XOFF}, 8 bits
        por carácter y un bit de stop, con la que funcionaba sin problemas.
      \item[Diferentes pulsos por segundo:] \hfill \\
        Un lado a 9600 baudios y otro a 14400. Funcionaba también sin
        ningún problema (probablemente de casualidad).
      \item[Diferente paridad:] \hfill \\
        Un lado con paridad par y otro con paridad impar. Funcionaba bien,
        probablemente porque \emph{PuTTY} no controlará los errores de paridad,
        por extraño que parezca.
      \item[Diferente tamaño de carácter:] \hfill \\
        Un lado con 5 bits y otro con 8: Tuvo el efecto esperado, no pudiendo
        enviar nada valioso entre ambos terminales.
      \item[Diferente tamaño de carácter (2):] \hfill \\
        Un lado con 7 bits y otro con 8: La terminal configurada con 8 bits podía
        enviar caracteres \texttt{ASCII} a la otra, aunque el octavo bit se perdía.

        En la otra dirección se enviaban caracteres no imprimibles, como se podía
        esperar.
    \end{description}
  \paragraph{Dato curioso}
    Para enviar caracteres especiales pudimos usar:
    \mint{bash}|echo "Hola \x00" > /dev/ttyS0|

\subsection{Modifique la configuración del puerto serie entre 8 y 7 bits de
            datos}
\subsubsection{Justifique lo que ocurre al transmitir en ambos casos la
               letra \texttt{ñ}}
  Con 8 bits la ñ se transmite sin problema, ya que la representación de dicho
  carácter en la codificación que estábamos usando (\texttt{UTF-8}) ocupa 8
  bits.

  Sin embargo con 7 bits no ocurre lo mismo, el bit más significativo (que
  resulta ser 1 en dicha codificación) se pierde, haciendo que llegue un
  carácter diferente al destinatario.
\subsection{Pruebe los programas de ejemplo síncrono y asíncrono para
            Windows y Linux}
\subsubsection{Comentar el functionamiento de los ejemplos facilitados}
  \paragraph{Linux}
    \begin{description}
      \item[\texttt{./chat}] \hfill \\
        Envía el texto de forma asíncrona al otro extremo del puerto serie,
        haciendo que se vea reflejado en el acto en la pantalla.

        No se manejan los caracteres de control como el backspace y el carácter
        nulo, aunque debería ser trivial hacerlo.
      \item[\texttt{./frase}] \hfill \\
        Envía el texto de forma síncrona en bloques de 80 caracteres.
    \end{description}
  \paragraph{Windows}
    El funcionamiento de los programas es idéntico al de Linux.
\subsubsection{Especifique los parámetros de configuración del puerto serie
               en cada uno de los ejemplos e indique también la/s línea/s de
               código en la que se configuran}
  \paragraph{Linux}
    En ambos ejemplos se configura el puerto mediante la función
    \texttt{serie\_abrir}, con el puerto directamente sacado de los argumentos,
    y la velocidad de 9600 baudios.

    Dentro de esa función (archivo \texttt{serielin/serielin.c:30}) se
    configura:
    \begin{itemize}
      \item Tamaño de carácter 8 (línea 42).
      \item Sin paridad (línea 43).
      \item Tamaño de buffer 1 (línea 46).
      \item Sin timeout de lectura (línea 47).
      \item La velocidad la pasada como argumento (50 y 51).
      \item Otro porrón de implicaciones que se basan en el valor de
        determinadas contstantes (modo no canónico, etc.) (líneas 44 y 45).
      \item Los parámetros se guardan en la línea 64.
    \end{itemize}

  \paragraph{Windows}
    La configuración del puerto serie se realiza a través de la función
    \texttt{serie\_abrir}, localizada en el archivo \texttt{seriewin.c}
    (línea 27)
    \begin{itemize}
      \item Apertura del puerto (línea 32).
      \item Obtención de la configuración anterior (línea 49).
      \item Configuración de los baudios (línea 55).
      \item Configuración del número de bits (línea 56).
      \item Configuración de la paridad (líneas 59 – 65).
      \item Configuración de los bits de stop (líneas 68 – 72).
      \item Hacer que no sea sensible al pin DSR (línea 74).
      \item Configurar para que se active el DTR cuando se abra el puerto
        (línea 75).
      \item Configurar para que no se observe el pin DSR para hacer control
        de flujo (línea 76).
      \item Guardado de la configuración (línea 85).
      \item Configurar los timeouts a 0 (líneas 91 – 104).
    \end{itemize}
\subsubsection{Comentar brevemente el API de Windows y Linux para
               configurar el puerto serie, y cómo se lee y escribe.}
Las APIs tanto de Windows como de Linux son similares en cuanto a que tratan
al puerto de serie como un fichero normal, pero a partir de ahí la cosa cambia
radicalmente:

La interfaz de Linux (\texttt{termios}) es la estándar de POSIX, aunque su
diseño puede ser discutible y está ligado al inicio de las comunicaciones entre
ordenadores (obtener una terminal en otro ordenador).

La de windows es una API mucho más genérica, que encaja con el resto del diseño
de windows NT.

\subsubsection{¿En qué se diferencia el modelo síncrono del asíncrono?}
En el modelo síncrono las lecturas son bloqueantes, mientras que en el modelo
asíncrono la lectura se lleva a cabo cuando el estado del puerto de serie
cambia (\texttt{SIGIO} en Linux, \texttt{CommEvent} en windows), y la ejecución
es no lineal.

\subsubsection{¿Qué ocurre si cruzamos el programa de ejemplo asíncrono para
               Windows con el de Linux? Justifique la respuesta.}
En principio debería de funcionar sin problemas (modulo cambios de codificación
de caracteres), gracias a la estandarización de la lectura/escritura en los
puertos de serie.

\subsection{Cree un fichero de texto con nombre serie.ini como el que figura
            a continuación:}
\begin{minted}[frame=lines]{ini}
[Configuracion]
Puerto=COM1
Velocidad=9600
BitsDatos=7
BitsParada=2
Paridad=Paridad par
\end{minted}
\subsubsection{Modifique y adjunte el programa de ejemplo asíncrono para
               Windows de forma que cargue los datos de configuración del
               puerto serie de este fichero. Utilice las funciones de Windows
               \texttt{GetPrivateProfileString} y \texttt{GetPrivateProfileInt}
               para recuperar directamente los valores del archivo de
               configuración.}
Módulo una corrección de errores más exhaustiva, y usar los valores explícitos
de Windows como \texttt{PTSTR} (donde nos aprovechamos de que no está definida
la macro \texttt{UNICODE}, y que por lo tanto un carácter ocupa 8 bits al igual
que el estándar de C), sería algo así:
\inputminted[frame=lines]{diff}{add-ini-config.diff}

\subsubsection{Verifique su correcto funcionamiento mostrando la configuración
               seleccionada en pantalla. Aporte las capturas de pantalla
               cambiando el contenido del fichero .ini y verificando el
               correcto funcionamiento del programa.}
% TODO

\end{document}
