\documentclass[10pt]{article}
\usepackage{lmodern}
\usepackage{amssymb,amsmath}
\usepackage{ifxetex,ifluatex}
\usepackage{fixltx2e} % provides \textsubscript
\usepackage{mathspec}
\defaultfontfeatures{Ligatures=TeX,Scale=MatchLowercase}
\newcommand{\euro}{€}
\usepackage[dvipsnames]{xcolor}
\usepackage{framed}
\usepackage{caption}
\usepackage{hyperref}
\usepackage[spanish]{babel}
\hypersetup{breaklinks=true,
            unicode=true,            colorlinks=true,
            citecolor=black,
            urlcolor=black,
            linkcolor=black,
            pdfborder={0 0 0}}
\urlstyle{same}  % don't use monospace font for urls
\usepackage[margin=1in]{geometry}
\usepackage{graphicx,grffile}
\makeatletter
\def\maxwidth{\ifdim\Gin@nat@width>\linewidth\linewidth\else\Gin@nat@width\fi}
\def\maxheight{\ifdim\Gin@nat@height>\textheight\textheight\else\Gin@nat@height\fi}
\makeatother
% Scale images if necessary, so that they will not overflow the page
% margins by default, and it is still possible to overwrite the defaults
% using explicit options in \includegraphics[width=\textwidth][width, height, ...]{}
% \setkeys{Gin}{width=\maxwidth,height=\maxheight,keepaspectratio}
\setlength{\parindent}{0pt}
\setlength{\parskip}{6pt plus 2pt minus 1pt}
\setlength{\emergencystretch}{3em}  % prevent overfull lines
\providecommand{\tightlist}{%
  \setlength{\itemsep}{0pt}\setlength{\parskip}{0pt}}
\setcounter{secnumdepth}{5}

\title{Guión OSPF\\\vspace{0.5em}{\large Redes de computadores II -- 2016}}
\author{Emilio Cobos Álvarez (70912324-N)}
\date{\today}

\newenvironment{answerfigure}{\begin{center}}{\end{center}}
\newenvironment{answer}{\color{Green}\begin{framed}}{\end{framed}}

% Redefines (sub)paragraphs to behave more like sections
\ifx\paragraph\undefined\else
\let\oldparagraph\paragraph
\renewcommand{\paragraph}[1]{\oldparagraph{#1}\mbox{}}
\fi
\ifx\subparagraph\undefined\else
\let\oldsubparagraph\subparagraph
\renewcommand{\subparagraph}[1]{\oldsubparagraph{#1}\mbox{}}
\fi

\begin{document}

\maketitle

\tableofcontents
\clearpage

\section{Nota para el corrector}

Siento no entregar la página en el documento propuesto para el guión, pero es
que trabajo mucho más lento con LibreOffice que con texto plano.

Iba a recuperar la numeración de los ejercicios, pero no me da tiempo. Siento
los inconvenientes y espero que puedas disculparme. Las respuestas siguen
resaltadas apropiadamente.

\section{Introducción}\label{introducciuxf3n}

En el fichero \emph{RIPOSPF.rar} está definida una red como la que se
muestra en la \protect\hyperlink{anchor-1}{Figura 1}. Descomprime el
fichero de configuración del escenario \emph{RIPOSPF.rar} en la carpeta
correspondiente de GNS3.

Arranca todas las máquinas y abre una consola con cada una de ellas
(orden \emph{consolas}). Los equipos PC1 y PC2 tienen rutas por defecto
a R1 y R4 respectivamente. Los \emph{routers} no tienen configurada
ninguna ruta, salvo la de las subredes a las que están directamente
conectados.

En los siguientes apartados se configurará OSPF en cada \emph{router}
para que las tablas de encaminamiento permitan alcanzar cualquier punto
de la red. Todos ellos estarán dentro de la misma área, luego los
configuraremos con identificador de área igual a 1.

\section{Activación de R1}\label{activaciuxf3n-de-r1}

Para observar los mensajes que envíe R1 cuando se active OSPF, arranca
\emph{wireshark} en todos los enlaces de R1. A continuación, configura
OSPF en el encaminador R1 para que su identificador de \emph{router}
sea la mayor de sus direcciones IP y exporte las rutas hacia las tres
redes a las que está conectado.

\begin{verbatim}
config t
router ospf 1
network 10.0.0.0 255.0.0.0 area 1
network 11.0.0.0 255.0.0.0 area 1
network 50.0.0.0 255.0.0.0 area 1
router-id 50.0.0.2
exit
exit
wr
\end{verbatim}

Activa la depuración de los mensajes ospf: debug ip ospf events

Espera un minuto aproximadamente e interrumpe las capturas.

Interrumpe también los mensajes de depuración: no \emph{debug ip ospf
events}

Analiza el comportamiento de R1 estudiando las capturas del tráfico y
los mensajes depuración para responder a las siguientes preguntas:

Observa los mensajes \emph{HELLO} que se envían al arrancar OSPF en R1
y analízalos utilizando \emph{wireshark}.

¿Cada cuánto tiempo se envían dichos mensajes? Observa si coincide
con el valor del campo \emph{HELLO Interval} de los mensajes.

\begin{answer}
En efecto, el valor coincide con el campo \emph{HELLO Interval}, y dicho
valor es 10 segundos.
\end{answer}

Comprueba que el campo \emph{Area ID} se corresponde con el
identificador de área que has configurado.


\begin{answer}
Sí, aunque el formato es el mismo que el típico para una dirección IPv4
(\texttt{0.0.0.1}).
\end{answer}

Comprueba que el identificador del \emph{router} se corresponde con
el que has configurado mirando el campo \emph{Source OSPF Router} de
la cabecera obligatoria de OSPF en los mensajes \emph{HELLO}.
Comprueba que este identificador es el mismo para los mensajes
enviados por cualquiera de las interfaces de R1, aunque los mensajes
se envíen con dirección IP origen diferente (cada mensaje llevará
como dirección IP origen la de la interfaz de red de R1 por la que
se envíe).

\begin{answer}
Sí, la id es \texttt{50.0.0.2}.
\end{answer}

Observa el valor de los campos \emph{DR} y \emph{BDR} en los
primeros mensajes \emph{HELLO}. ¿Qué ocurre con dichos campos
transcurridos 40 segundos después del primer mensaje \emph{HELLO}?
¿Por qué?

\begin{answer}
En los primeros mensajes el valor del \emph{Designated Router} y del
\emph{Backup Designated Router} es de \texttt{0.0.0.0}. Sin embargo,
tras pasar 40 segundos, el valor del \emph{Designated Router} cambia a
la IP del router en la interfaz correspondiente.

El router se convierte en el router designado porque es el único router
de la subred, y no tiene prioridad 0.
\end{answer}

¿Se observan en las capturas mensajes \emph{DB Description} o \emph{LS
Update}? ¿Por qué?

\begin{answer}
No, no hay ninguna, porque no hay más routers en la subred, y OSPF es lo
suficientemente ``listo'' como para saberlo y no enviar paquetes
repetidos.
\end{answer}

¿Debería haber aprendido alguna ruta R1? Compruébalo consultando la
tabla de encaminamiento mediante la orden \emph{show ip route}.
Incluye aquí la salida.

\begin{answer}
No, no ha podido aprender ninguna porque no hay ningún otro router en la
subred ``hablando'' OSPF.

\begin{answerfigure}
\includegraphics[width=\textwidth]{img/1-ospf-show-ip-route-r1.png}
\captionof{figure}{show ip route}
\end{answerfigure}
\end{answer}

Consulta la información de los vecinos que ha conocido R1 a través de
los mensajes \emph{HELLO} recibidos mediante: \emph{show ip ospf
neighbor}. Incluye y comenta la salida obtenida.


\begin{answer}
Como era de esperar, la salida del comando está vacía.

\begin{answerfigure}
\includegraphics[width=\textwidth]{img/1-ospf-show-ip-ospf-neighbor-r1.png}
\captionof{figure}{show ip ospf neighbor}
\end{answerfigure}
\end{answer}

Consulta la información de la base de datos de \emph{Router Link
States} de R1 con: \emph{show ip ospf database router}. Incluye y
comenta la salida obtenida\emph{.}

\begin{answer}
La primera línea nos dice la id del router (\texttt{50.0.0.2}), y su id
de proceso interno (\texttt{1}).

Se puede ver cómo aparece el número de interfaces, y las IDs y máscaras
de red de las subredes a las que está conectado.

\begin{answerfigure}
\includegraphics[width=\textwidth]{img/1-ospf-show-ip-ospf-database-router-r1.png}
\captionof{figure}{show ip ospf database router}
\end{answerfigure}
\end{answer}

Consulta la información de la base de datos de \emph{Network Link
States} de R1 con: \emph{show ip ospf database network}. Incluye y
comenta la salida obtenida.

\begin{answer}
Esta es la base de datos de los routers que conocen OSPF. Como era de
esperar el único router que aparece es él mismo.

\begin{answerfigure}
\includegraphics[width=\textwidth]{img/1-ospf-show-ip-ospf-database-network.png}
\captionof{figure}{show ip ospf database network}
\end{answerfigure}
\end{answer}

\section{Activación de R2}\label{activaciuxf3n-de-r2}

Para observar los mensajes que envíe R2 cuando se active OSPF, y los que
envíe R1 a consecuencia de la activación de R2, arranca \emph{wireshark}
en los enlaces de R2 con R1 y R3 y en el enlace de R1 con R5.

A continuación, configura OSPF en el encaminador R2 para que su
identificador de \emph{router} sea la mayor de sus direcciones IP y para
que exporte las rutas hacia las dos redes a las que está conectado.
Incluye aquí las órdenes:

\begin{verbatim}
config t
router ospf 1
network 10.0.0.0 255.0.0.0 area 1
network 20.0.0.0 255.0.0.0 area 1
router-id 20.0.0.1
exit
exit
wr
\end{verbatim}

Activa los mensajes de depuración. Incluye aquí la orden

\begin{answer}
\begin{verbatim}
debug ip ospf events
\end{verbatim}

\begin{answerfigure}
\includegraphics[width=\textwidth]{img/1-ospf-enable-r2.png}
\captionof{figure}{config and debug}
\end{answerfigure}
\end{answer}

Espera dos minutos aproximadamente e interrumpe las capturas. Analiza el
comportamiento de R1 y R2 estudiando las capturas de tráfico, los
mensajes de depuración, consultando el estado de OSPF y de la orden
\emph{show ip route ospf} en cada encaminador:

Observa la captura realizada entre R1 y R2 y responde a las siguientes
cuestiones:

¿Qué tipo de mensajes aparecen cuando R1 detecta la presencia de R2
y viceversa? ¿Cuál es su propósito? ¿Qué IP de destino llevan esos
mensajes?

\begin{answer}
Tras el primer mensaje \texttt{Hello} de \texttt{R2} al grupo multicast,
y tras haber recibido un Hello de \texttt{R1} por la misma vía,
\texttt{R1} y \texttt{R2} sincronizan sus bases de datos usando mensajes
\texttt{DB\ Description}.

Estos mensajes van vía unicast al router interesado. Tras esos mensajes
se realizan también \texttt{LS\ requests} (\emph{Link State Requests})
mutuas vía unicast, que son respondidas con los correspondientes
\texttt{LS\ updates} (también vía unicast).

Cuando los routers individuales reciben las updates y cambia la
información que almacenan, informan \textbf{al grupo multicast} de que
la información ha cambiado con un \texttt{LS\ Update}, y cuando reciben
la información vía multicast, envían la confirmación también vía
multicast (\texttt{LS\ Acknowledge}).

Si la información de su LSDB cambia por los datos contenidos en el
\texttt{LS\ Update}, otro \texttt{LS\ Update} se enviará al grupo
multicast.

\begin{answerfigure}
\includegraphics[width=\textwidth]{img/2-ospf-r1-connect-r2.png}
\captionof{figure}{captura relevante}
\end{answerfigure}
\end{answer}

Observa los mensajes \emph{LS Request} que envían R1 y R2. ¿Qué
\emph{LSA} pide cada uno al otro? ¿Qué IP de destino llevan estos
mensajes?

\begin{answer}
La IP de destino es la IP de la interfaz del router conectada
directamente, pero el \emph{Link State ID} es la ID del router, es decir
la IP mayor del router en este caso.

Por ejemplo, en el primer \emph{LS Request} de R1 a R2, tenemos un
mensaje unicast de la dirección \texttt{10.0.0.1} a la dirección
\texttt{10.0.0.2}, pero como el router OSPF de origen tenemos
\texttt{50.0.0.2}, y como \emph{Link State ID} tenemos la
\texttt{20.0.0.1}, es decir, las IDs de R1 y R2 respectivamente.

De hecho, el paquete tiene \texttt{TTL=1}, por lo que no llegaría si la
IP de destino estuviera en otra subred.
\end{answer}

Observa el primer mensaje \emph{LS Update} que envía R1. Comprueba
que se corresponde con el \emph{LS Request} enviado por R2.
Comprueba cómo se corresponde su contenido con lo almacenado en la
base de datos de R1 analizada en el apartado anterior. Observa sus
campos para ver si este mensaje incluye la información de que R1 ha
descubierto a R2 como vecino. ¿Crees que la información contenida en
este mensaje deberá cambiar próximamente? ¿Por qué? Observa el campo
\emph{LS Age} del anuncio que viaja en el mensaje, y explica su
valor.

\begin{answer}
Podemos comprobar cómo se corresponde con el de la Request de R2 porque
podemos ver cómo tiene los campos \texttt{Link\ State\ ID} y
\texttt{Advertising\ Router} iguales a la correspondiente Request
(\texttt{50.0.0.2}).

No aparece la información de que R1 ha descubierto a R2, aparecen las
tres conexiones directas de R1.

El contenido de este mensaje no tendrá que cambiar, porque es un mensaje
LSA del router, es decir, sólo anuncia las interfaces configuradas, no
qué routers son vecinos o similar\footnotemark{}, y en R1 ya constaba la
conexión directa con esas tres redes.

Sobre el campo \emph{LS Age}, es cuánto tiempo ha pasado desde que el
LSA correspondiente fue generado. En este caso eran 400 segundos, que
era aproximadamente el tiempo que había pasado desde que encendí el
escenario.
\end{answer}
\footnotetext{Lo siento, tenía que dejar esto aquí:
https://www.youtube.com/watch?v=aPtr43KHBGk}

\begin{answerfigure}
\includegraphics[width=\textwidth]{img/2-ospf-ls-update-r1-r2.png}
\captionof{figure}{ls update R1 to R2}
\end{answerfigure}

Observa el primer mensaje \emph{LS Update} que envía R2. Comprueba
que se corresponde con el \emph{LS Request} enviado por R1. Observa
sus campos para ver si este mensaje incluye la información de que R2
ha descubierto a R1 como vecino. ¿Crees que la información contenida
en este mensaje deberá cambiar próximamente? ¿Por qué? Observa el
campo \emph{LS Age} del anuncio que viaja en el mensaje, y explica
su valor.

\begin{answer}
Se puede comprobar de manera análoga, el Link State ID es
\texttt{20.0.0.1}, que es el correspondiente al request de R1.

Igualmente, al ser un mensaje Router-LSA no tiene que contener los datos
de que es vecino de R1, sino las interfaces configuradas. En estas no se
incluye la red \texttt{20.0.0.0} todavía (supongo que por no tener un
DR?), así que tendrá que cambiar e incluirlo.

El campo LS Age en este caso es de 9 segundos, que entra dentro de lo
esperado.
\end{answer}

Observa el segundo y tercer mensajes \emph{LS Update} que envía R1.
¿Responden a algún \emph{LS Request} previo? ¿Por qué se envían?
¿Qué información contienen? Observa el campo \emph{LS Age} de los
anuncios que viajan en los mensajes, y explica su valor.

\begin{answer}
No, no responden a un LS Request previo, sino que se son enviados al
grupo multicast.

El segundo contiene las interfaces de R1 (router-LSA), con la interfaz
\texttt{10.0.0.1} marcada como \emph{transit} en vez de como \emph{stub}
(lo que quiere decir que mi respuesta a la penúltima pregunta era
incorrecta). Esto quiere decir que ahora el tráfico entre routers OSPF
puede pasar por esa red. Es enviado porque la información de la LSDB de
R1 ha cambiado.

El tercero es un network-LSA, que contiene los dos routers OSPF
conectados a la red e información sobre ellos (\texttt{50.0.0.2} y
\texttt{20.0.0.1}). Es enviado porque la información de la topología de
la red ha cambiado y R1 es el DR.

El campo LS Age es 1, porque la información acaba de ser incorporada a
la LSDB.
\end{answer}

Observa el segundo mensaje \emph{LS Update} que envía R2. ¿Responde
a algún \emph{LS Request} previo? ¿Por qué se envía? ¿Qué
información contiene? Observa el campo \emph{LS Age} del anuncio que
viaja en el mensaje, y explica su valor.

\begin{answer}
No, ese segundo mensaje no responde a ningún request, sino que se envía
porque la LSDB de R2 ha cambiado dado que ahora se ha dado cuenta de que
estaba conectado a la red \texttt{20.0.0.0}.

El LS Age es 1 por esa misma razón, es información recién añadida a la
LSDB.
\end{answer}

¿Por qué razón R2 no envía ningún mensaje \emph{Network-LSA}?

\begin{answer}
Porque no es el DR de la subred.
\end{answer}

Observa los mensajes \emph{LS Acknowledge}. Mira su contenido para
comprobar a qué LSAs asienten.

\begin{answer}
Podemos comprobar tanto el número de secuencia como el Link State ID
(ojo, \textbf{dentro del LSA Header}) para asegurarnos a qué LSA están
asintiendo. Estarán asintiendo a todos los mensajes provenientes del
Link State ID y con menor número de secuencia que los especificados en
el paquete.
\end{answer}

Pasados 40 segundos del arranque de R2, ¿qué ocurre con los campos
\emph{DR} y \emph{BDR} de los mensajes \emph{HELLO} que
intercambian?

\begin{answer}
Que R2 se convierte en el BDR de la subred.
\end{answer}

Observa la captura realizada en R5:

Explica por qué no aparecen los mensajes \emph{LS Update} que crea
R1 y envía a R2.


\begin{answer}
Porque R1 está directamente conectado con R2, y son mensajes unicast.
\end{answer}

Explica por qué no aparecen los mensajes \emph{LS Update} que crea
R2 y envía a R1, y R1 debería propagar por inundación.

\begin{answer}
Porque son mensajes con TTL=1, y por lo tanto son deshechados cuando van
a ser enviados.
\end{answer}

Observa la captura realizada en R3:

Explica por qué no aparecen los mensajes \emph{LS Update} que crea
R2 y envía a R1.

\begin{answer}
Por la misma razón, hay conexión directa entre R1 y R2.
\end{answer}

Explica por qué no aparecen los mensajes \emph{LS Update} que crea
R1 y envía a R2, y R2 debería propagar por inundación.

\begin{answer}
Porque R2 no es el DR, y además los mensajes tienen TTL=1
\end{answer}

¿Deberían haber aprendido alguna ruta R2 y R1? Compruébalo consultando
la tabla de encaminamiento en ambos encaminadores mediante la orden
\emph{show ip route ospf}. Incluye aquí la salida. Comprueba la
métrica de cada ruta y a través de qué \emph{router} se alcanza.

\begin{answer}
Sí, R2 debería haber aprendido a ir a \texttt{50.0.0.0} y a
\texttt{11.0.0.0}, y R1 debería haber aprendido a ir a
\texttt{20.0.0.0}.

Podemos ver como es lo que esperábamos:

\begin{answerfigure}
\includegraphics[width=\textwidth]{img/2-ospf-show-ip-route-ospf-r1-and-r2.png}
\captionof{figure}{show ip route ospf en R1 y R2}
\end{answerfigure}
\end{answer}

Consulta la información de los vecinos que ha conocido cada
encaminador a través de los mensajes \emph{HELLO} mediante: \emph{show
ip ospf neighbor}. Incluye la salida.

\begin{answer}
\begin{answerfigure}
\includegraphics[width=\textwidth]{img/2-ospf-show-ip-ospf-neighbor-r1-and-r2.png}
\captionof{figure}{show ip ospf neighbor en R1 y R2}
\end{answerfigure}
\end{answer}

Consulta en cada encaminador la información de las bases de datos de
\emph{Router Link States} y de \emph{Network Link States} mediante:
\emph{show ip ospf database router} y \emph{show ip ospf database
network} respectivamente. Comprueba que la información mostrada
coincide con el contenido de los últimos \emph{LS Update} enviados por
los encaminadores.

\begin{answer}
Podemos ver como los router link states son idénticos en ambos routers,
y coincide con la información mandada en los últimos Router LS Update de
R1 y R2.

\begin{answerfigure}
\includegraphics[width=\textwidth]{img/2-ospf-show-ip-ospf-database-router-r1-and-r2.png}
\captionof{figure}{show ip ospf database router}
\end{answerfigure}

Igualmente, podemos ver cómo los Network Link States son iguales,
contienen tanto a R1 como R2, y también se corresponden con el último
network LSU de R1 al grupo multicast:

\begin{answerfigure}
\includegraphics[width=\textwidth]{img/2-ospf-database-network-r1-r2.png}
\captionof{figure}{show ip ospf database router}
\end{answerfigure}

Nótese que el número de secuencia no se corresponde porque había pasado
algo de tiempo desde la captura, pero los datos no han cambiado.
\end{answer}

Consulta un resumen de las bases de datos en cada encaminador con:
\emph{show ip ospf database}. Incluye y comenta la salida obtenida.

\begin{answer}
\begin{answerfigure}
\includegraphics[width=\textwidth]{img/2-ospf-show-ip-ospf-database-r1-r2.png}
\captionof{figure}{show ip ospf database}
\end{answerfigure}

Se puede comprobar como las bases de datos son idénticas (todos los
routers pueden ver la topología de su área).

Los routers actualmente son R1 y R2 (con sus correspondientes IDs), y
las redes son la \texttt{10.0.0.0}, cuyo DR es el \texttt{50.0.0.2}.
\end{answer}

\section{Activación de R3 y R4}\label{activaciuxf3n-de-r3-y-r4}

Para observar los mensajes que envíen R3 y R4 cuando activen OSPF, y los
que envíe R2 a consecuencia de la activación de R3 y R4, arranca
\emph{wireshark} en los enlaces entre R1 y R2, entre R2 y R3 y R3 con
R4.

Configura OSPF en R3 y en R4. Para tratar de arrancarlos a la vez
prepara las ordenes necesarias en un fichero de texto para copiar y
pegar en cada uno de los encaminadores. Escribe aquí las órdenes
necesarias.

\begin{answer}
R3:

\begin{verbatim}
config t
router ospf 1
network 20.0.0.0 255.0.0.0 area 1
network 30.0.0.0 255.0.0.0 area 1
router-id 30.0.0.1
exit
exit
wr
\end{verbatim}

R4:

\begin{verbatim}
config t
router ospf 1
network 30.0.0.0 255.0.0.0 area 1
network 40.0.0.0 255.0.0.0 area 1
network 41.0.0.0 255.0.0.0 area 1
router-id 41.0.0.1
exit
exit
wr
\end{verbatim}

Nótese que podríamos haber usado \texttt{passive-interface} con R4 (y
con R1). No debería de afectar, pero dada la falta de tiempo, y que la
ID de R4 sería (no obligatoriamente, pero\ldots{}) la interfaz por la
que se usaría \texttt{passive-interface}, no he querido arriesgarme.
Cuando haya más tiempo lo pruebo.
\end{answer}

Analiza el comportamiento de los encaminadores estudiando las capturas
con \emph{wireshark}, los mensajes de depuración, consultando el estado
de OSPF y de la orden show ip \emph{route ospf} en cada encaminador:

Trata de suponer los valores de \emph{DR} y \emph{BDR} en las subredes
20.0.0.0/8 y 30.0.0.0/8. Comprueba si tus suposiciones son ciertas.
Comprueba en los mensajes \emph{HELLO} de la captura en R3 cómo se ha
producido la elección de \emph{DR} y \emph{BDR} al arrancar R3 y R4 a
la vez.

\begin{answer}
En la subred \texttt{20.0.0.0}, se debería de mantener el DR, y usar a
\texttt{20.0.0.1} como BDR, ya que el router \texttt{20.0.0.1} ya estaba
designado anteriormente, y por lo tanto el router \texttt{20.0.0.2}
quedaría como BDR.

En la subred \texttt{30.0.0.0}, debería de elegirse al router
\texttt{30.0.0.2} como DR, por ser el mayor de los dos, y a
\texttt{30.0.0.1} como BDR.

Podemos comprobar como estaba en lo cierto:

\begin{answerfigure}
\includegraphics[width=\textwidth]{img/3-ospf-designated-router.png}
\captionof{figure}{DR y BDR}
\end{answerfigure}
\end{answer}

En la captura del enlace R3-R4 observa el intercambio de mensajes
\emph{LS Update} que se produce mientras arrancan R3 y R4.

\begin{answer}
Podemos comprobar cómo R3 envía a R4 un total de 5 LSAs (3 router, 2
network) vía unicast.

Las tres router-LSA corresponden con los LSA de \texttt{20.0.0.1}, con
los de \texttt{30.0.0.1} (los propios), y con los de \texttt{50.0.0.2},
que contienen toda la topología de red que conoce R3 en el momento.

Sin embargo, R4 envía otro LS Update, pero sólo con Router-LSA, que es
lo que él conoce en el momento, que son los stubs de las redes a donde
está conectado.

Tras recibirlo R3 lanza al grupo multicast un Router-LSA, con él ya como
DR y la red \texttt{30.0.0.0} en el estado \texttt{Transit} en vez de
Stub, y posteriormente (también al grupo multicast) un Network-LSA con
las dos IDs como adyacentes (\texttt{30.0.0.1} y \texttt{41.0.0.1}).
\end{answer}

En la captura del enlace R2-R3 observa el intercambio de mensajes
\emph{LS Update} que se produce mientras arrancan R3 y R4. Observa
también en dicha captura los mensajes \emph{LS Update} que R3 envía
por inundación de los recibidos por el de R4. Indica cómo puedes saber
si un \emph{LS Update} lo ha originado el encaminador que lo envía o
está siendo propagado por inundación.

\begin{answer}
Se puede comprobar si un LS Update ha sido propagado mirando el
\texttt{Advertising\ Router}. En este caso por ejemplo, se ven varios
updates con las redes \texttt{41.0.0.0}, \texttt{40.0.0.0} y
\texttt{30.0.0.0} que vienen por parte de \texttt{R4}.
\end{answer}

Antes de examinar la captura en el enlace de R1-R2 trata de suponer
qué tipos de mensaje aparecerán en ella. Comprueba tus suposiciones.

\begin{answer}
Supongo que llegarán mensajes propagados tanto de R3 como de R4, con las
nuevas rutas, y los routers ya existentes necesitarán sincronizar las
LSDB.

Podemos ver como ese es el caso, efectivamente recibiendo
incrementalmente las actualizaciones por parte de R3 y R4, por parte de
R2 claro (\texttt{10.0.0.2}), y con los correspondientes acklowdedges de
R1.

\begin{answerfigure}
\includegraphics[width=\textwidth]{img/3-ospf-enable-r3-r4-in-r1-r2.png}
\captionof{figure}{Captura de R2 a R1}
\end{answerfigure}
\end{answer}

Trata de suponer qué modificaciones se habrán realizado en las tablas
de encaminamiento de cada \emph{router}. Observa las tablas de
encaminamiento para verificar tus suposiciones.

\begin{answer}
Ahora mismo todos los routers tendrían que tener todas las subredes como
accesibles dentro de sus tablas de encaminamiento, con más o menos
saltos (por ejemplo, R1 tendrá las redes \texttt{40.0.0.0} y
\texttt{41.0.0.0} con la métrica más alta, 4 si no me equivoco, debería
ser 1 menos que RIP).

\begin{answerfigure}
\includegraphics[width=\textwidth]{img/3-ospf-show-ip-route.png}
\captionof{figure}{Tablas de rutas de los 4 routers}
\end{answerfigure}
\end{answer}

Consulta la información de los vecinos que ha conocido cada
encaminador a través de los mensajes \emph{HELLO}.

\begin{answer}
Podemos hacerlo fácilmente con \texttt{show\ ip\ ospf\ neighbor}:

\begin{answerfigure}
\includegraphics[width=\textwidth]{img/3-ospf-show-ip-ospf-neighbor.png}
\captionof{figure}{Vecinos de los routers}
\end{answerfigure}
\end{answer}

Consulta en cada encaminador la información de las bases de datos de
\emph{Router Link States} y de \emph{Network Link States}. Comprueba
que la información mostrada coincide con el contenido de los últimos
\emph{LS Update} enviados por los encaminadores.

\begin{answer}
No se incluye la captura aquí por ser demasiado grande, y la información
se puede encontrar abajo, pero se puede comprobar cómo en efecto la
correspondencia existe, y coincide con los últimos mensajes LS Update
salvo por los cambios en números de secuencia dada la diferencia de
tiempo.
\end{answer}

Consulta el resumen de las bases de datos en cada encaminador.

\begin{answer}
\begin{answerfigure}
\includegraphics[width=\textwidth]{img/3-ospf-show-ip-ospf-database.png}
\captionof{figure}{}
\end{answerfigure}
\end{answer}

Tras haber arrancado OSPF en los encaminadores R1, R2, R3 y R4, PC1 y
PC2 deberían tener conectividad IP. Compruébalo con las órdenes
\emph{ping} y trace (incluye aquí su salida).

\begin{answer}
\begin{answerfigure}
\includegraphics[width=\textwidth]{img/3-ospf-ping-trace.png}
\captionof{figure}{Ping y trace de PC1 a PC2}
\end{answerfigure}
\end{answer}

\section{Reconfiguración de rutas: activación y desactivación de
R5}\label{reconfiguraciuxf3n-de-rutas-activaciuxf3n-y-desactivaciuxf3n-de-r5}

Deja lanzado el \emph{ping} de PC1 a PC2 (\emph{ping 40.0.0.10 -t}), y
reinicia OSPF en R1 \emph{(clear ip ospf 1 process}). ¿Se ha producido
pérdida de paquetes? ¿Por qué? Comprueba lo que ha sucedido con las
capturas de tráfico necesarias. Compara lo sucedido para esta misma
situación en la práctica de RIP.

\begin{answer}
Podemos comprobar cómo \textbf{sí se pierden paquetes} (unos 10
aproximadamente), porque se tiene que volver a rehacer toda la
reelección de DBR y sincronización de la LSDB.

Podemos ver en el debug log de OSPF cómo el mayor período de pérdida de
paquetes es hasta que recibe intercambian los DBD (\emph{DataBase
Descriptor}), es decir, sincronizan las bases de datos, y recibe el
primer LS Update de R2.

A diferencia de RIP, donde todo el estado estaba disponible dentro de la
red multicast, y era proveído casi instantáneamente por los vecinos, en
OSPF, dada la mayor complejidad del protocolo, la información tarda
bastante más en estar disponible.
\end{answer}

Realiza los cambios necesarios para que la ruta seguida por los
datagramas IP que envía PC1 a PC2 vayan por la ruta
\(PC1 \rightarrow R1 \rightarrow R5 \rightarrow R4 \rightarrow PC2\),
y para que los que envía PC2 a PC1 sigan la ruta
\(PC2 \rightarrow R4 \rightarrow R5 \rightarrow R1 \rightarrow PC1\).
Para realizar este apartado no podrás añadir o eliminar manualmente
rutas en las tablas de encaminamiento. Mirando la tabla de
encaminamiento de R1, observa y apunta el número de segundos que
aproximadamente tarda en aprender R1 la nueva ruta.

\begin{answer}
Al igual que en la anterior práctica, con configurar R5 adecuadamente
(en este caso con OSPF) debería de ser suficiente. Vamos a comprobarlo
experimentalmente:

\begin{verbatim}
config t
router ospf 1
network 50.0.0.0 255.0.0.0 area 1
network 40.0.0.0 255.0.0.0 area 1
router-id 50.0.0.1
exit
exit
wr
\end{verbatim}

La ruta la ha aprendido relativamente rápidamente (estaba copiando el
tiempo del primer \texttt{recv\ hello} (\texttt{00:49:14.251}), y cuando
he querido poner la tabla de rutas (un par de segundos) ya la conocía.
\end{answer}

Comprueba que se está utilizando dicha ruta a través de la orden
\emph{trace}. Comprueba las rutas y sus métricas en las tablas de
encaminamiento de cada encaminador. Compara este valor con el anotado
para esta misma situación en la práctica de RIP.

\begin{answer}
\begin{answerfigure}
\includegraphics[width=\textwidth]{img/4-ospf-ping-trace-pc1-pc2-with-r5.png}
\captionof{figure}{Ping y trace con R5 configurado}
\end{answerfigure}

Podemos comprobar cómo \textbf{las métricas son exactamente las de RIP más 1}.

\begin{answerfigure}
\includegraphics[width=\textwidth]{img/4-ospf-router-db-r5.png}
\captionof{figure}{Bases de datos de los routers}
\end{answerfigure}
\end{answer}

Deja lanzado en PC1 un \emph{ping} hacia PC2. Lanza las capturas de
tráfico necesarias para explicar qué sucede cuando se interrumpe la
ejecución de OSPF en el encaminador R5 (utiliza la orden \emph{no
router ospf }). Podrás observar con la orden \emph{show ip route} que
ahora R5 no conoce rutas aprendidas por OSPF. Tampoco exporta
información de vecinos hacia otros encaminadores.

¿Deja de funcionar el \emph{ping} de PC1 a PC2? ¿durante cuánto
tiempo? (fíjate en el número de secuencia \emph{icmp\_seq}, éste
aumenta con cada paquete enviado cada segundo).

\begin{answer}
Podemos comprobar como sí se pierde tráfico, durante 3 segundos
aproximadamente, pero nada excesivamente importante.

Este poco tiempo de downtime se puede explicar porque el router 5 manda
un paquete LS Update a R1 cuando es desconfigurado, que R1 propaga
inmediatamente. R1 puede entonces re-enrutar los paquetes por R2
inmediatamente\footnotemark{}
\begin{answerfigure}
  \includegraphics[width=\textwidth]{img/4-ospf-disable-r5.png}
  \captionof{figure}{Captura de tráfico al deshabilitar R5}
\end{answerfigure}
\end{answer}

\footnotetext{¿Es en cuanto realiza R1 el ack del LS Update de R2? Es lo que
  parece, ya que se ven un par de paquetes lanzados hacia R5 tras el LS ack de
  R1, pero podría ser que R1 tardara en re-evaluar las rutas.}.

Observa durante este periodo, en el que no está funcionando R5, la
tabla de encaminamiento de R1 y R4 y su lista de vecinos. Describe lo
que ocurre. Muestra aquí la evolución de estas tablas. ¿Cuánto tiempo
tardan R1 y R4 en olvidar las rutas por R5 y aprender las nuevas? ¿por
qué? ¿Cuánto tiempo tarda R5 en desaparecer de la lista de vecinos de
R1 y R4? ¿por qué? Ayúdate de las capturas del tráfico para explicar
lo sucedido y contestar adecuadamente a las preguntas anteriores.
Compara lo sucedido para esta misma situación en la práctica de RIP.

\begin{answer}
Parte de esta respuesta está respondida arriba. Tras otra comprobación,
esa rápida propagación del LS Update mandado por el mismo R5 hacia R1 es
la causante del rápido olvido de esas rutas y esos vecinos.

Se propaga el LS Update de las rutas de R5 por toda la red, y es cuando
recibe R1 otro update por parte de R2, con las nuevas rutas que él tiene
para llegar a las redes \texttt{40.0.0.0} y \texttt{50.0.0.0} cuando
comienza a enviar paquetes por R2 para llegar a \texttt{PC2}.

Nótese que en dicho LS Update de R2 no viene información explícita sobre
cómo llegar a \texttt{41.0.0.0}, sino sólo de las rutas que han
cambiado. R1 ya sabe el estado de los links de R4 (y que es el único
router por el que llegar a \texttt{PC2}, por lo que no le hace falta esa
información de nuevo.

Podemos comprobar que, gracias a tener esa actualización cuando cambia
(que sería equivalente a tener triggered updates en RIP cuando se
desconfigura), el tiempo durante el cual los datos de la tabla de rutas
son erróneos es mucho menor.

Sería interesante comprobar cuánto se tarda si el fallo fuera por un
apagón (y por lo tanto ese último LS Update no se realizara). No creo
que tenga tiempo antes de entregar esta práctica, pero probablemente lo
pruebe con más calma.
\end{answer}

Interrumpe el \emph{ping} y comprueba la ruta que están siguiendo los
mensajes intercambiados entre PC1 y PC2 con \emph{trace}. Incluye aquí
la salida

\begin{answer}
Podemos comprobar como pasa por R2:

\begin{answerfigure}
\includegraphics[width=\textwidth]{img/4-ospf-trace-pc1-pc2-without-r5.png}
\captionof{figure}{Captura del trace de PC1 a PC2}
\end{answerfigure}
\end{answer}

Por último, vuelve a configurar de nuevo \emph{OSPF} en R5. Observa
cómo cambian las tablas de encaminamiento en R1 y R4 y apenas se
interrumpe el \emph{ping}. Comprueba de nuevo cuál es ahora la ruta
que están siguiendo los mensajes intercambiados entre PC1 y PC2 con
\emph{trace}. Observa y apunta el número de segundos que
aproximadamente tarda en aprenderse de nuevo la ruta a través de R5,
mirando continuamente la tabla de encaminamiento de R1. Mira también
los números de secuencia de los \emph{icmps} del \emph{ping}, y fíjate
si alguno se pierde mientras se cambia de la ruta antigua a la ruta
nueva. Compara estos datos con los observados para esta misma
situación en la práctica de RIP.

\begin{answer}
Podemos ver cómo no se pierde ni un sólo paquete. Las tablas de
encaminamiento están actualizadas en cuestión de un par de segundos, y
la ruta que se sigue pasa a ser R5 prácticamente instantáneamente (desde
que se estabiliza todo de nuevo y se reciben los correspondientes
acknowledges).

Este tiempo es ligeramente más lento que RIP. Ya he comentado esto con
más detalle arriba, pero resumiendo, en RIP no hace falta tanta
sincronización como en OSPF (en cuanto un vecino se daba cuenta de que
había un vecino nuevo le mandaba los datos correspondientes).

\begin{answerfigure}
\includegraphics[width=\textwidth]{img/4-ospf-reconnect-r5.png}
\captionof{figure}{Capturas relevantes}
\end{answerfigure}
\end{answer}

\end{document}
