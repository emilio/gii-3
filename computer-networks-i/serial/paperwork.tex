\documentclass[a4paper]{article}

\usepackage{fontspec}
\usepackage{titlesec}
\usepackage{float}
\usepackage{enumitem}
\usepackage{tabularx}
\usepackage{hyperref}
\usepackage{wrapfig}
\usepackage{minted}
\usepackage{chngcntr}
\usepackage[spanish]{babel}
\usepackage[cm]{fullpage}

\counterwithout{subsection}{section}

\usemintedstyle{friendly}
\setminted{breaklines,frame=lines}

\setlength{\parindent}{2em}
\setlength{\parskip}{.5em}

\titleclass{\section}{top}
\newcommand\sectionbreak{\clearpage}

\titleformat{\section}[hang]{\normalfont\bfseries\Large}{Bloque \thesection: }{0pt}{}
\titleformat{\subsubsection}[hang]{\normalfont\it}{\thesubsubsection) }{10pt}{}
\titleformat{\paragraph}[hang]{\normalfont\bfseries}{\thesubsubsection) }{0}{}
\titlespacing{\paragraph}{1em}{0pt}{0pt}

\renewcommand{\thesection}{\Alph{section}}
\renewcommand{\thesubsubsection}{\alph{subsubsection}}
\renewcommand{\thesubsection}{\arabic{subsection}}

\setlist[description]{leftmargin=\dimexpr2.5\parindent,itemindent=\dimexpr-1\parindent}

\title{Práctica 1: El puerto serie}
\author{Emilio Cobos Álvarez (B1) \\ Anna Elena Chesnais (B1) \\
        Cristina García González (B1)}

\begin{document}
\maketitle

\tableofcontents

% Disable listoffigures clearpage
\begingroup
\let\clearpage\relax

\listoffigures

\endgroup

\section{El cable de módem nulo}

\subsection{Construya un cable de módem nulo con el fin de comunicar dos
            equipos por el puerto serie (recomendada versión completa).}

\subsubsection{Use el programa \textit{PuTTY} para comprobar dicha comunicación.
               ¿Qué parámetros se pueden modificar en el puerto serie?.
               ¿Qué valores son posibles en cada uno de ellos?.
               Pruebe distintas configuraciones (al menos dos) y describa lo
               que ocurre al teclear caracteres en una y otra consola.
               ¿Se puede seleccionar el conjunto de caracteres que el terminal
               remoto está utilizando?. ¿Para qué sirve esto?.}

  \paragraph{Parámetros modificables y valores aceptados}
    \begin{description}
        \item[Puerto:]
          Cualquier nombre de archivo que represente un puerto de serie.
        \item[Pulsos por segundo (baudios):]
          Cualquiera de los valores estándares que enumeramos a continuación: \\
            110,
            300\footnote{Valor mínimo con el que pudimos transmitir datos},
            600, 1200, 2400, 4800, 9600, 14400, 19200, 28800, 38400,
            56000, 57600, 115200, 128000,
            153600 \footnote{El valor máximo que nuestra conexión soportaba},
            230400, 256000, 460800 y 921600.
        \item[Bits de datos de un carácter:]
          5, 6, 7 u 8 bits.
        \item[Bits de stop:]
          1, 1.5 ó 2.
        \item[Paridad:]
          Ninguna, par o impar.
        \item[Control de flujo:]
          Ninguno, \texttt{XON/XOFF} o \texttt{RTS/CTS}.
      \end{description}
  \paragraph{Configuraciones probadas}
    \begin{description}
      \item[Configuración por defecto:] \hfill \\
        9600 baudios, sin paridad, control de flujo \texttt{XON/XOFF}, 8 bits
        por carácter y un bit de stop, con la que funcionaba sin problemas.
      \item[Diferentes pulsos por segundo:] \hfill \\
        Un lado a 9600 baudios y otro a 14400. Funcionaba también sin
        ningún problema (probablemente de casualidad).
      \item[Diferente paridad:] \hfill \\
        Un lado con paridad par y otro con paridad impar. Funcionaba bien,
        probablemente porque \textit{PuTTY} no controlará los errores de
        paridad, por extraño que parezca.
      \item[Diferente tamaño de carácter:] \hfill \\
        Un lado con 5 bits y otro con 8: Tuvo el efecto esperado, no pudiendo
        enviar nada valioso entre ambos terminales.
      \item[Diferente tamaño de carácter (2):] \hfill \\
        Un lado con 7 bits y otro con 8: La terminal configurada con 8 bits podía
        enviar caracteres \texttt{ASCII} a la otra, aunque el octavo bit se
        perdía.

        En la otra dirección se enviaban caracteres no imprimibles, como se podía
        esperar.
    \end{description}
  \paragraph{Dato curioso}
    Para enviar caracteres especiales pudimos usar:
    \mintinline{bash}{echo "Hola \x00" > /dev/ttyS0}

\subsection{Modifique la configuración del puerto serie entre 8 y 7 bits de
            datos}

\subsubsection{Justifique lo que ocurre al transmitir en ambos casos la
               letra \texttt{ñ}}

  Con 8 bits la ñ se transmite sin problema, ya que la representación de dicho
  carácter en la codificación que estábamos usando (\texttt{UTF-8}) ocupa 8
  bits.

  Sin embargo con 7 bits no ocurre lo mismo, el bit más significativo (que
  resulta ser 1 en dicha codificación) se pierde, haciendo que llegue un
  carácter diferente al destinatario.
\subsection{Pruebe los programas de ejemplo síncrono y asíncrono para
            Windows y Linux}

\subsubsection{Comentar el functionamiento de los ejemplos facilitados}

  \paragraph{Linux}
    \begin{description}
      \item[\texttt{./chat}] \hfill \\
        Envía el texto de forma asíncrona al otro extremo del puerto serie,
        haciendo que se vea reflejado en el acto en la pantalla.

        No se manejan los caracteres de control como el backspace y el carácter
        nulo, aunque debería ser trivial hacerlo.
      \item[\texttt{./frase}] \hfill \\
        Envía el texto de forma síncrona en bloques de 80 caracteres.
    \end{description}
  \paragraph{Windows}
    El funcionamiento de los programas es idéntico al de Linux.

\subsubsection{Especifique los parámetros de configuración del puerto serie
               en cada uno de los ejemplos e indique también la/s línea/s de
               código en la que se configuran}

  \paragraph{Linux}
    En ambos ejemplos se configura el puerto mediante la función
    \texttt{serie\_abrir}, con el puerto directamente sacado de los argumentos,
    y la velocidad de 9600 baudios.

    Dentro de esa función (archivo \texttt{serielin/serielin.c:30}) se
    configura:
    \begin{itemize}
      \item Tamaño de carácter 8 (línea 42).
      \item Sin paridad (línea 43).
      \item Tamaño de buffer 1 (línea 46).
      \item Sin timeout de lectura (línea 47).
      \item La velocidad la pasada como argumento (50 y 51).
      \item Otro porrón de implicaciones que se basan en el valor de
        determinadas contstantes (modo no canónico, etc.) (líneas 44 y 45).
      \item Los parámetros se guardan en la línea 64.
    \end{itemize}

  \paragraph{Windows}
    La configuración del puerto serie se realiza a través de la función
    \texttt{serie\_abrir}, localizada en el archivo \texttt{seriewin.c}
    (línea 27)
    \begin{itemize}
      \item Apertura del puerto (línea 32).
      \item Obtención de la configuración anterior (línea 49).
      \item Configuración de los baudios (línea 55).
      \item Configuración del número de bits (línea 56).
      \item Configuración de la paridad (líneas 59 – 65).
      \item Configuración de los bits de stop (líneas 68 – 72).
      \item Hacer que no sea sensible al pin DSR (línea 74).
      \item Configurar para que se active el DTR cuando se abra el puerto
        (línea 75).
      \item Configurar para que no se observe el pin DSR para hacer control
        de flujo (línea 76).
      \item Guardado de la configuración (línea 85).
      \item Configurar los timeouts a 0 (líneas 91 – 104).
    \end{itemize}

\subsubsection{Comentar brevemente el API de Windows y Linux para
               configurar el puerto serie, y cómo se lee y escribe.}

Las APIs tanto de Windows como de Linux son similares en cuanto a que tratan
al puerto de serie como un fichero normal, pero a partir de ahí la cosa cambia
radicalmente:

La interfaz de Linux (\texttt{termios}) es la estándar de POSIX, aunque su
diseño puede ser discutible y está ligado al inicio de las comunicaciones entre
ordenadores (obtener una terminal en otro ordenador).

La de Windows es una API mucho más genérica, que encaja con el resto del diseño
de \textit{Windows NT}.

\subsubsection{¿En qué se diferencia el modelo síncrono del asíncrono?}

En el modelo síncrono las lecturas son bloqueantes, mientras que en el modelo
asíncrono la lectura se lleva a cabo cuando el estado del puerto de serie
cambia (\texttt{SIGIO} en Linux, \texttt{CommEvent} en Windows), y la ejecución
es no lineal.

\subsubsection{¿Qué ocurre si cruzamos el programa de ejemplo asíncrono para
               Windows con el de Linux? Justifique la respuesta.}

En principio debería de funcionar sin problemas (modulo cambios de codificación
de caracteres), gracias a la estandarización de la lectura/escritura en los
puertos de serie.

\subsection{Cree un fichero de texto con nombre serie.ini como el que figura
            a continuación:}

\begin{minted}{ini}
[Configuracion]
Puerto=COM1
Velocidad=9600
BitsDatos=7
BitsParada=2
Paridad=Paridad par
\end{minted}

\subsubsection{Modifique y adjunte el programa de ejemplo asíncrono para
               Windows de forma que cargue los datos de configuración del
               puerto serie de este fichero. Utilice las funciones de Windows
               \texttt{GetPrivateProfileString} y \texttt{GetPrivateProfileInt}
               para recuperar directamente los valores del archivo de
               configuración.}

Se migró el código a \textit{Visual Studio} porque \textbf{el compilador de
\textit{DEV-C++} en Windows 10 padecía de errores internos}.

De todas formas la migración no fue excesivamente complicada, salvo por ciertos
problemas con la configuración del proyecto (\textit{Visual Studio} define la
constante \texttt{UNICODE} por defecto, lo que cambia el ancho del carácter que
usa la API de \textit{Windows}, por ejemplo), o de restricciones del compilador
(fuerza determinados castings, el uso de \texttt{\_getch} vs \texttt{getch},
etc).

Los cambios aplicados al archivo son los siguientes (\texttt{etc/chat-4.c}):

\inputminted{diff}{etc/add-ini-config.diff}

\subsubsection{Verifique su correcto funcionamiento mostrando la configuración
               seleccionada en pantalla. Aporte las capturas de pantalla
               cambiando el contenido del fichero .ini y verificando el
               correcto funcionamiento del programa.}

\begin{figure}[H]
  \begin{center}
    \includegraphics[width=0.6\textwidth]{screenshots/ini-1.png}
    \includegraphics[width=0.6\textwidth]{screenshots/ini-2.png}
    \includegraphics[width=0.6\textwidth]{screenshots/ini-3.png}
    \caption{Algunos ejemplos de ejecución del programa con el archivo
             \texttt{.ini} al lado}
  \end{center}
\end{figure}

\section{El control de flujo}

\subsection{Durante la transferencia de información entre dispositivos, puede
            ocurrir que la velocidad de procesado de uno de ellos sea inferior a
            la del otro, con lo que se podría perder parte de dicha información
            si no se utilizase algún mecanismo de arbitraje. Este mecanismo es
            precisamente el control de flujo. En la interfaz RS-232, existen dos
            modalidades: el control de flujo hardware y el control de flujo
            software.}

\subsubsection{Con el apoyo de la herramienta Hércules describe el proceso de
               control de flujo por hardware al enviar datos de un equipo a
               otro. En este caso el control de flujo se realiza mediante las
               líneas \texttt{RTS} y \texttt{CTS} del RS-232. Si lo consideras
               necesario solicita el Mini Tester al profesor para confirmar
               visualmente la activación/desactivación de estas líneas.}

Cuando el puerto está cerrado, el pin \texttt{RTS} está en negativo.

Cuando el puerto del otro ordenador se intenta conectar, el \texttt{CTS} pasa a
positivo, ya que el otro ordenador a activado su \texttt{RTS} (que está
cruzado), lo que indica que el otro terminal está listo.

Si conectamos el puerto, nuestro \texttt{RTS} pasa también a positivo (y por lo
tanto el \texttt{CTS} del otro terminal).

Cuando ambos pines en ambos dispositivos están en positivo, podemos transmitir
datos.

Si uno de los dos terminales usa \texttt{RTS}/\texttt{CTS} y el otro no, el
primer terminal no enviará datos, pero sí de trata de recibirlos (pone el pin
\texttt{RTS} a positivo, pero no recibe la señal de \texttt{CLS}, por lo que no
envía). Sin embargo el otro terminal sí envía, y los datos llegan al primero.

\subsubsection{Con el apoyo de la herramienta \textit{Hércules}describe el
               proceso de control de flujo por software al enviar datos de un
               equipo a otro. En este caso el control de flujo se hace mediante
               las líneas de datos por medio del envío de caracteres
               predeterminados (\texttt{XOFF} para detener y \texttt{XON} para
               reanudar la transmisión).}

Con \texttt{XON} y \texttt{XOFF}, tanto el pin \texttt{RTS} como el \texttt{CTS}
están en negativo (ya que no se usan). El control de flujo se realiza por
software (cuando se recibe el carácter \texttt{XOFF} se deja de transmitir,
hasta que se vuelva a recibir el carácter \texttt{XON}.

\section{El control de errores de la interfaz serie}

\subsection{Realice las modificaciones necesarias en el programa resultante
            del punto 4 (la configuración del puerto serie se lee del fichero
            serie.ini) de tal forma que en la recepción se puedan detectar los
            posibles errores. Consulte ``Serial Status" del manual ``MSDN -
            Serial Communications". Adjunte el proyecto con el código fuente
            modificado.}

Sólo se ha usado \texttt{printf} para detectar el error, ya que el manejo del
mismo no se especificaba en el enunciado.

Nótese también que se ha usado \texttt{GetCommMask} y \texttt{SetCommMask} para
evitar modificar el archivo \texttt{seriewin.c}.

Los cambios aplicados con respecto al otro ejercicio son los siguientes
(\texttt{etc/chat-6.c}):

\inputminted{diff}{etc/error-detection.diff}

\subsection{Debido a que el entorno en el que se desarrollan las prácticas
            posee una tasa de error muy baja es difícil detectar los posibles
            errores. Por esta razón, y siempre sobre el código del ejercicio
            anterior, se pide lo siguiente:}

\subsubsection{Modifique las configuraciones del puerto serie en emisor y
               receptor para que se puedan producir un errores de paridad
               (\texttt{CE\_RXPARITY}). Especifique y justifique los valores
               escogidos para la configuración.}

\begin{figure}[H]
  \begin{center}
    \includegraphics[width=0.6\textwidth]{screenshots/parity-error.png}
    \caption{Configuración utilizada para provocar errores de paridad:
             Símplemente se escogió diferentes tipos de paridad en ambos
             terminales (par en uno e impar ene el otro).}
  \end{center}
\end{figure}


\subsubsection{Modifique las configuraciones del puerto serie en emisor y
               receptor para que se puedan producir errores de formato
               (\texttt{CE\_FRAME}).Especifique y justifique los valores
               escogidos para la configuración.}

\begin{figure}[H]
  \begin{center}
    \includegraphics[width=0.6\textwidth]{screenshots/frame-error.png}
    \caption{Configuración utilizada para provocar errores de formato: Se
             utilizaron diferentes tamaños en ambos terminales tanto de bits de
             datos como de parada, con el fin de forzar el envío de cadenas de
             bits de longitud no esperada por el receptor}
  \end{center}
\end{figure}

\subsubsection{Modifique el código para que se pueda obtener en el receptor una
               condición de línea a 0 ó sin tensión (\texttt{CE\_BREAK}). Pista:
               Revise el API de Windows para el manejo del puerto serie.}

Nuestro código originalmente ya podía detectar los \texttt{CE\_BREAK}, así que
no ha habido necesidad de modificarlo de nuevo.

\subsubsection{Modifique el código para que no comience el proceso de recepción
               hasta que no se haya pulsado la tecla \texttt{ESC}. Especifique
               los pasos para provocar un error de sobreescritura del buffer de
               entrada (\texttt{CE\_RXOVER}) con el nuevo código.}

% TODO

\section{Terminal por el puerto serie}

\subsection{Una de las aplicaciones prácticas del puerto serie es la de
            permitir controlar un sistema Unix de forma remota conectado
            directamente por dicho puerto. Precisamente ésta es la manera más
            típica que existe para la gestión de muchos elementos de red (p. ej:
            conmutadores, enrutadores, pasarelas, etc.).}
\subsubsection{Localice e investigue la orden getty para conseguir sacar un
               terminal por el puerto serie con una velocidad de 9600 baudios y
               con una emulación para el terminal vt100.}

Al usar \texttt{getty} se abre el puerto serie y se permite crear una terminal
remota desde un cliente.

La orden ejecutada, como se puede ver en las capturas de pantalla del directorio
\texttt{screenshots}, fue:

\begin{minted}{bash}
getty ttyS0 9600 vt100
\end{minted}

Nótese que el argumento \texttt{vt100} es completamente innecesario, ya que es
por defecto.

La consola se queda leyendo de la entrada estándar indefinidamente, hasta que el
cliente remoto cierra la sesión.

\begin{figure}[H]
  \begin{center}
    \includegraphics[width=0.6\textwidth]{screenshots/getty.png}
    \caption{Orden utilizada, tras que el cliente haya cerrado la sesión.}
  \end{center}
\end{figure}

\begin{figure}[H]
  \begin{center}
    \includegraphics[width=0.6\textwidth]{screenshots/getty-putty.png}
    \caption{Cliente usando putty para conectarse.}
  \end{center}
\end{figure}
\end{document}
