\documentclass[10pt]{article}
\usepackage{lmodern}
\usepackage{amssymb,amsmath}
\usepackage{ifxetex,ifluatex}
\usepackage{fixltx2e} % provides \textsubscript
\usepackage{mathspec}
\defaultfontfeatures{Ligatures=TeX,Scale=MatchLowercase}
\newcommand{\euro}{€}
\usepackage[dvipsnames]{xcolor}
\usepackage{framed}
\usepackage{caption}
\usepackage{hyperref}
\usepackage[spanish]{babel}
\hypersetup{breaklinks=true,
            unicode=true,            colorlinks=true,
            citecolor=black,
            urlcolor=black,
            linkcolor=black,
            pdfborder={0 0 0}}
\urlstyle{same}  % don't use monospace font for urls
\usepackage[margin=1in]{geometry}
\usepackage{graphicx,grffile}
\makeatletter
\def\maxwidth{\ifdim\Gin@nat@width>\linewidth\linewidth\else\Gin@nat@width\fi}
\def\maxheight{\ifdim\Gin@nat@height>\textheight\textheight\else\Gin@nat@height\fi}
\makeatother
% Scale images if necessary, so that they will not overflow the page
% margins by default, and it is still possible to overwrite the defaults
% using explicit options in \includegraphics[width=\textwidth][width=\textwidth][width, height, ...]{}
% \setkeys{Gin}{width=\maxwidth,height=\maxheight,keepaspectratio}
\setlength{\parindent}{0pt}
\setlength{\parskip}{6pt plus 2pt minus 1pt}
\setlength{\emergencystretch}{3em}  % prevent overfull lines
\providecommand{\tightlist}{%
  \setlength{\itemsep}{0pt}\setlength{\parskip}{0pt}}
\setcounter{secnumdepth}{5}

\title{Guión RIP\\\vspace{0.5em}{\large Redes de computadores II -- 2016}}
\author{Emilio Cobos Álvarez (70912324-N)}
\date{21 de Abril de 2015}

\newenvironment{exercise}{}{}
\newenvironment{answerfigure}{\begin{center}}{\end{center}}
\newenvironment{answer}{\color{Green}\begin{framed}}{\end{framed}}

\begin{document}
\maketitle

\tableofcontents
\clearpage

\section{Nota para el corrector}

Siento no entregar la página en el documento propuesto para el guión, pero es
que trabajo mucho más lento con LibreOffice que con texto plano.

Iba a recuperar la numeración de los ejercicios, pero no me da tiempo. Siento
los inconvenientes y espero que puedas disculparme. Las respuestas siguen
resaltadas apropiadamente.

\section{Funcionamiento básico}\label{funcionamiento-buxe1sico}

\begin{exercise}
En el fichero \emph{RIPOSPF.rar} está definida una red como la que se
muestra en la Figura 1. Descomprime el fichero de configuración del
escenario \emph{RIPOSPF.rar} en la carpeta correspondiente de GNS3.

Arranca todas las máquinas y abre una consola con cada una de ellas
(orden \emph{consolas}). Los equipos PC1 y PC2 tienen rutas por defecto
a R1 y R4 respectivamente. Compruébalo con la orden \emph{show o show
ip} (Incluye aquí esa información).
\end{exercise}

\begin{answer}
\begin{answerfigure}
\includegraphics[width=\textwidth]{img/show-pc1.png}
\captionof{figure}{Show PC1}
\end{answerfigure}

\begin{answerfigure}
\includegraphics[width=\textwidth]{img/show-pc2.png}
\captionof{figure}{Show PC2}
\end{answerfigure}
\end{answer}

\begin{exercise}
Los \emph{routers} no tienen configurada ninguna ruta, salvo la de las
subredes a las que están directamente conectados. Compruébalo con la
orden \emph{show ip route} (Incluye aquí esa información).
\end{exercise}

\begin{answer}
\begin{answerfigure}
\includegraphics[width=\textwidth]{img/show-r1.png}
\captionof{figure}{Show R1}
\end{answerfigure}

\begin{answerfigure}
\includegraphics[width=\textwidth]{img/show-r2.png}
\captionof{figure}{Show R2}
\end{answerfigure}

\begin{answerfigure}
\includegraphics[width=\textwidth]{img/show-r3.png}
\captionof{figure}{Show R3}
\end{answerfigure}

\begin{answerfigure}
\includegraphics[width=\textwidth]{img/show-r4.png}
\captionof{figure}{Show R4}
\end{answerfigure}

\begin{answerfigure}
\includegraphics[width=\textwidth]{img/show-r5.png}
\captionof{figure}{Show R5}
\end{answerfigure}
\end{answer}

\begin{exercise}
En los siguientes apartados se configurará RIP en los \emph{routers}
para que las tablas de encaminamiento permitan alcanzar cualquier punto
de la red.

Para observar los mensajes que envíe R1 cuando se active RIP, arranca
\emph{wireshark} en todos los enlaces de R1. A continuación configura
RIP en R1 para que exporte las rutas de las tres redes a las que está
conectado.
\end{exercise}

\begin{answer}
\begin{verbatim}
config t
router rip
version 2
network 10.0.0.0
network 11.0.0.0
network 50.0.0.0
no auto-summary
exit
exit
wr
\end{verbatim}

\begin{answerfigure}
\includegraphics[width=\textwidth]{img/r1-rip-config.png}
\captionof{figure}{R1 RIP config}
\end{answerfigure}
\end{answer}

\begin{exercise}
Activa la depuración de los mensajes rip: \emph{debug ip rip}

Espera un minuto aproximadamente e interrumpe las capturas.

Interrumpe también los mensajes de depuración: no \emph{debug ip rip}

Analiza el comportamiento de R1 estudiando las capturas del tráfico y
los mensajes de depuración para responder a las siguientes preguntas:

Observa los mensajes REQUEST que se envían al arrancar \emph{RIP} en R1
y analiza su contenido. ¿Son iguales en todas las interfaces? ¿Para qué
se utilizan? ¿Qué rutas viajan en estos mensajes? ¿Quién responde a
estos mensajes?
\end{exercise}

\begin{answer}
Sí, todos los mensajes \texttt{REQUEST} son iguales para todas las
interfaces, se utilizan para que otros routers que estén potencialmente
escuchando en el grupo multicast de RIP v2 manden las rutas tan pronto
como puedan.

Obviamente no hay respuesta (no hay ningún otro router configurado con
RIP). Esos mensajes son mensajes a dicho grupo multicast
\texttt{224.0.0.9}, aunque con \texttt{TTL=2}. No obstate sólo llegan a
los routers vecinos, al ser un paquete multicast local.

Sobre el porqué el ttl es 2, he encontrado alguna información
interesante\footnotemark{}
\end{answer}
\footnotetext{https://learningnetwork.cisco.com/thread/40817, http://stackoverflow.com/questions/9745429/why-is-eigrp-and-rip-uses-ip-ttl-of-2-cisco}.

\begin{exercise}
Observa los mensajes RESPONSE que envía R1 periódicamente a través de
cada una de sus interfaces. ¿Son iguales en todas las interfaces? ¿Qué
rutas viajan en estos mensajes?
\end{exercise}

\begin{answer}
No, no todos los mensajes son iguales para todas las interfaces, por
ejemplo, no anuncia que puede conectarse a la red \texttt{11.0.0.0} por
esa misma red.

Así, al hub son anunciadas las redes \texttt{50.0.0.0} y
\texttt{10.0.0.0}, a la red \texttt{10.0.0.0} son anunciadas las redes
\texttt{11.0.0.0} y \texttt{50.0.0.0}, y a la red \texttt{50.0.0.0} son
anunciadas las redes \texttt{10.0.0.0} y \texttt{11.0.0.0}.
\end{answer}

\begin{exercise}
¿Debería haber aprendido alguna ruta R1? Compruébalo consultando la
tabla de encaminamiento mediante la orden \emph{show ip route}.
\end{exercise}

\begin{answer}
No, ya que no ha obtenido ninguna respuesta de ninguno de los otros
routers, así que todavía sólo conocerá las redes a las que está
directamente conectado.
\end{answer}

\begin{exercise}
Ejecuta en R1 la orden \emph{show ip protocols}. ¿Qué información nos
facilita? Adjunta y comenta la salida obtenida.
\end{exercise}

\begin{answer}
Nos facilita múltiples aspectos de la configuración del router, entre
ellos cómo está configurado con RIP, y con qué parámetros (tiempo de
olvido, periodo actualización\ldots{}).

\textbf{Nota}: Esta captura se ha perdido, por eso se adjunta una hecha con el
router re-configurado en mi ordenador personal.

\begin{answerfigure}
  \includegraphics[width=\textwidth]{img/post-show-ip-protocols-r1-rip.png}
  \captionof{figure}{Show IP protocols en R1}
\end{answerfigure}
\end{answer}

\begin{exercise}
¿Para qué sirve la orden
\texttt{passive-interface\ \textless{}interface\textgreater{}}? ¿Sería
útil configurar en R1 alguna de sus interfaces de este modo?. Incluye
aquí la orden.
\end{exercise}

\begin{answer}
Suprime las actualizaciones del routing en una interfaz, es decir, que
no manda actualziaciones a esa interfaz.

Es útil, por ejemplo, configurar la interfaz \texttt{FastEthernet} así
(la asociada al hub), ya que sabemos que no hay ningún otro router
asociado a esa interfaz.

La orden es: \texttt{passive-interface\ FastEthernet\ 0/0}
\end{answer}

\begin{exercise}
Utiliza esta orden en el resto de routers del escenario en los que sea
necesario.
\end{exercise}

\begin{exercise}
Ve configurando y arrancando a continuación RIP en los encaminadores R2,
R3 y R4, de uno en uno: primero en R2, luego en R3 y finalmente en R4.
Cada vez que configures uno de esos encaminadores realiza las capturas
que estimes conveniente, estudiándolas junto con los mensajes de
depuración y responde a las siguientes cuestiones:

Incluye las órdenes de configuración.
\end{exercise}

\begin{answer}
Se omiten las partes repetitivas para no sobrecargar la hoja:

R2:

\begin{verbatim}
  network 20.0.0.0
  network 10.0.0.0
\end{verbatim}

R3:

\begin{verbatim}
network 20.0.0.0
network 30.0.0.0
\end{verbatim}

R4:

\begin{verbatim}
network 30.0.0.0
network 40.0.0.0
network 41.0.0.0
passive-interface FastEthernet 0/0
\end{verbatim}
\end{answer}

\begin{exercise}
Comprueba el envío de mensajes REQUEST. ¿Existe algún mensaje de
RESPONSE a esos REQUEST? ¿Por qué?
\end{exercise}

\begin{answer}
En las interfaces donde no hay ningún router RIP conectado (y no son
passive-interface) sí hay un mensaje REQUEST, pero sólo obtendrán
RESPONSE a través de las interfaces que sí tienen un router RIP. No
obstante, esos RESPONSE no están dirigidos directamente a ese router,
sino al grupo multicast.

Aparte, una vez guardas la configuración, el router envía un mensaje
RESPONSE con las redes al grupo multicast\footnotemark{}
\end{answer}
\footnotetext{Como dato curioso, al configurar uno de los routers se me olvidó
  escribir ``version 2'' antes de los comandos ``network \ldots{}'', y se vió un
  mensje del tipo REQUEST del protocolo RIPv1 (obviamente sin respuesta), aunque
  fue corregido rápidamente.}.

\begin{exercise}
Comprueba la tabla de encaminamiento del encaminador recién arrancado,
así como las tablas de encaminamiento del resto de los encaminadores
para ver cómo se van propagando las rutas. Explica el proceso de
aprendizaje de rutas apoyándote en las capturas realizadas y en los
mensajes de depuración.
\end{exercise}

\begin{answer}
El proceso de aprendizaje no es especialmente complicado. El router
inicialmente sólo sabe las rutas a las que está conectado directamente.

En cuanto configuras RIP, el router envía un mensaje REQUEST a cada
interfaz, que es rápidamente contestado con la tabla de rutas de los
routers \textbf{directamente conectados} a esa interfaz.

Si esas tablas no estuvieran actualizadas, una vez pasaran 30 segundos
se volverían a mandar, presumiblemente actualizadas esta vez, y el
router sobrescribiría las entradas correspondientes con la información
más nueva.

Si los routers tuvieran triggered updates, no habría que esperar a otro
ciclo para actualizarlas, sino que en cuanto una de las tablas de los
routers vecinos cambiara, el cambio sería propagado instantáneamente.

El router también envía un RESPONSE nada más es configurado con RIP con
las rutas a las que está conectado directamente.
\end{answer}

\begin{exercise}
Comprueba la métrica de cada una de las rutas aprendidas. Detállalas a
continuación.
\end{exercise}

\begin{answer}
Las métricas son las esperadas:

\begin{itemize}
\tightlist
\item
  R1: Métrica 1 a \texttt{20.0.0.0}, 2 a \texttt{30.0.0.0}, y 3 a
  \texttt{40.0.0.0} y \texttt{41.0.0.0}.
\item
  R2: Métrica 1 a \texttt{11.0.0.0}, \texttt{50.0.0.0} y
  \texttt{30.0.0.0}, y 2 a \texttt{40.0.0.0} y \texttt{41.0.0.0}.
\item
  R3: Métrica 1 a \texttt{10.0.0.0}, \texttt{40.0.0.0} y
  \texttt{41.0.0.0}, métrica 2 a \texttt{50.0.0.0} y \texttt{11.0.0.0}.
\item
  R4: Métrica 1 a \texttt{20.0.0.0}, 2 a \texttt{10.0.0.0}, y 3 a
  \texttt{50.0.0.0} y \texttt{11.0.0.0}.
\end{itemize}

\textbf{Nota}: Esta captura se ha perdido, abajo se puede ver una con los datos
esperados tomada en mi ordenador particular tras reconfigurar la red.

\begin{answerfigure}
  \includegraphics[width=\textwidth]{img/post-metrics-rip.png}
  \captionof{figure}{Métricas de R1, R2, R3 y R4}
\end{answerfigure}
\end{answer}

\begin{exercise}
¿La implementación de RIP que utilizan los routers de Cisco está
empleando el mecanismo \emph{Split Horizon} o el mecanismo \emph{Split
Horizon + Poison Reverse}? ¿Cómo lo sabes?
\end{exercise}

\begin{answer}
Estos routers utilizan tanto split horizon como poison reverse, ya que
reenvían las rutas inalcanzables con métrica 16 a los otros routers.

El ejemplo se verá después, cuando se desconecte R5.
\end{answer}

\begin{exercise}
Tras haber arrancado RIP en los encaminadores R1, R2, R3 y R4, PC1 y PC2
deberían tener conectividad IP. Compruébalo con las órdenes \emph{ping}
y trace (incluye aquí su salida).
\end{exercise}

\begin{answer}
\begin{answerfigure}
\includegraphics[width=\textwidth]{img/ping-pc1-pc2.png}
\captionof{figure}{Ping PC1 a PC2}
\end{answerfigure}

\begin{answerfigure}
\includegraphics[width=\textwidth]{img/trace-pc1-pc2-without-r5.png}
\captionof{figure}{Trace PC1 a PC2}
\end{answerfigure}
\end{answer}

\begin{exercise}
Deja lanzado el \emph{ping} de PC1 a PC2 (\emph{ping 40.0.0.10 -t}), y
borra la tabla de rutas de R1 (\texttt{clear\ ip\ route\ *}). ¿Se ha
producido pérdida de paquetes? ¿Por qué? Comprueba lo que ha sucedido
con las capturas de tráfico necesarias.
\end{exercise}

\begin{answer}
No, no se ha producido pérdida de paquetes, ya que nada más se limpia la
tabla de rutas se manda un REQUEST a los routers adyacentes, que es casi
inmediatamente respondido.

\begin{answerfigure}
\includegraphics[width=\textwidth]{img/ping-pc1-clear-ip-route.png}
\captionof{figure}{Clear ip route during ping}
\end{answerfigure}
\end{answer}

\begin{exercise}
A continuación, realiza los cambios necesarios para que la ruta seguida
por los datagramas IP que envía PC1 a PC2 sea
\(PC1 \rightarrow R1 \rightarrow R5 \rightarrow R4 \rightarrow PC2\), y
para que los que envía PC2 a PC1 sigan la ruta
\(PC2 \rightarrow R4 \rightarrow R5 \rightarrow R1 \rightarrow PC1\).
Para realizar este apartado no podrás añadir o eliminar manualmente
rutas en las tablas de encaminamiento. Describe las acciones realizadas.
\end{exercise}

\begin{answer}
Lo único necesario que hay que hacer para ello es configurar RIP en R5
de manera análoga a como se ha hecho en el resto de los routers, y
esperar a que las rutas se propaguen.

Es fácil de ver cómo esas rutas son las rutas con menor número de saltos
entre sendos PC.
\end{answer}

\begin{exercise}
Mirando la tabla de encaminamiento de R1, observa y apunta el número de
segundos que aproximadamente tarda en aprenderse la nueva ruta. ¿Por
qué?
\end{exercise}

\begin{answer}
El tiempo de aprendizaje de dicha ruta es prácticamente instantáneo,
dado el RESPONSE que manda R5 nada más es configurado.
\end{answer}

\begin{exercise}
Comprueba que se está utilizando dicha ruta a través de la orden
\emph{trace}.
\end{exercise}

\begin{answer}
\begin{answerfigure}
\includegraphics[width=\textwidth]{img/trace-pc1-pc2.png}
\captionof{figure}{Trace PC1 a PC2}
\end{answerfigure}
\end{answer}

\begin{exercise}
Comprueba las rutas y sus métricas en las tablas de encaminamiento de
cada encaminador. Inclúyelas aquí. Comenta la salida.
\end{exercise}

\begin{answer}
Las métricas son, otra vez, las esperadas, y gracias a RIP, hemos
conseguido que ningún paquete de una red a otra tarde más de dos saltos
en llegar al destinatario.

\begin{answerfigure}
\includegraphics[width=\textwidth]{img/show-ip-route-rip.png}
\captionof{figure}{Tablas de rutas RIP}
\end{answerfigure}
\end{answer}

\begin{exercise}
¿Por qué ruta deberían ir los datagramas IP que envíe PC1 a la dirección
\texttt{30.0.0.2}? Justifica la respuesta. Comprueba tu respuesta
utilizando \emph{trace}.
\end{exercise}

\begin{answer}
Podría ir tanto por la red \texttt{10.0.0.0} como por la red
\texttt{50.0.0.0}, ya que la métrica es la misma para ambas redes (hay
el mismo número de saltos).

Usando \texttt{trace} comprobamos que la ruta escogida es la
\texttt{50.0.0.0}.

\begin{answerfigure}
\includegraphics[width=\textwidth]{img/trace-pc1-30.0.0.2.png}
\captionof{figure}{Trace PC1 a 30.0.0.2}
\end{answerfigure}
\end{answer}

\begin{exercise}
Con la misma red del escenario, y con los 5 encaminadores con RIP
activado, ¿Podrían haber seguido otra ruta los datagramas IP PC1 a la
dirección 30.0.0.2? ¿Cómo actúan los routers CISCO cuando reciben una
ruta con igual métrica que la que ya tienen en su tabla de rutas?
\end{exercise}

\begin{answer}
Se puede comprobar arriba cómo los routers CISCO almacenan ambas rutas
con la misma métrica, en vez de descartar la menos actualizada.

En teoría, según la documentación\footnotemark{}
se usa para hacer load balancing, es decir, distribuir paquetes
equitativamente.
\end{answer}
\footnotetext{http://www.cisco.com/c/en/us/support/docs/ip/border-gateway-protocol-bgp/5212-46.html},

\section{Eliminación de rutas}\label{eliminaciuxf3n-de-rutas}
\begin{exercise}
El objetivo de este apartado es observar lo que ocurre cuando se
interrumpe \emph{RIP} en R5. Se estudiará, en particular, el
comportamiento de los encaminadores R1 y R4.
\end{exercise}

\begin{answer}
Asegúrate de que los 5 encaminadores tienen arrancado \emph{RIP}.
Comprueba la ruta que están siguiendo los mensajes intercambiados entre
PC1 y PC2 con \emph{trace}.

\begin{answerfigure}
\includegraphics[width=\textwidth]{img/trace-pc1-pc2.png}
\captionof{figure}{Trace PC1 a PC2}
\end{answerfigure}
\end{answer}

\begin{exercise}
\begin{enumerate}
\def\labelenumi{\arabic{enumi}.}
\tightlist
\item
  Deja en ejecución en PC1 un \emph{ping} hacia PC2.
\item
  Para ver los mensajes RIP que envían R1 y R4, arranca \emph{wireshark}
  en sus interfaces g3/0 y activa los mensajes de depuración.
\item
  A continuación, interrumpe la ejecución de \emph{RIP} en el
  encaminador R5 utilizando la orden \emph{no router rip}. Podrás
  observar con la orden \emph{show ip route} que ahora R5 no conoce
  rutas aprendidas por RIP. Tampoco exporta rutas hacia otros
  encaminadores.
\item
  Observarás que el \emph{ping} de PC1 a PC2 deja de funcionar durante
  un buen rato. Observa durante este periodo en el que no está
  funcionando RIP en R5, las entradas de las tablas de encaminamiento de
  R1 y de R4. Observa la evolución de la columna de tiempo de las
  distintas entradas. ¿Qué entradas no reinician la cuenta cada 30
  segundos? ¿Por qué? Observa el valor de tiempo de esas entradas.
  Presta especial atención cuanto el valor de la columna de tiempo de
  algunas entradas de R1 y R4 se acerquen a 3 minutos ¿qué ocurrirá
  después?
\end{enumerate}
\end{exercise}

\begin{answer}
Las entradas correspondientes a R5 no se actualizan cada 30 segundos,
porque no se reciben esas actualizaciones (R5 no está usando RIP).

Una vez pasa de los 180 segundos, la ruta se da por muerta y se elimina.
Este valor está configurado y se puede ver usando
\texttt{show\ ip\ protocols}.
\end{answer}

\begin{exercise}
Poco después el \emph{ping} entre PC1 y PC2 habrá vuelto a funcionar.
¿Por qué? Interrúmpelo y, mirando los valores del \texttt{icmp\_seq}
apunta el número de segundos que aproximadamente ha estado sin funcionar
el \emph{ping} debido a que aún no se había olvidado la ruta a través de
R5. Comprueba la ruta que están siguiendo los mensajes intercambiados
entre PC1 y PC2 con \emph{trace}. Incluye aquí la salida.
\end{exercise}

\begin{answer}
El ping vuelve a funcionar porque R1 empieza a usar la ruta por la red
\texttt{10.0.0.0}, al dar por muerta la ruta antigua.

Ahora la ruta es la alternativa, por R2 y R3.

\begin{answerfigure}
\includegraphics[width=\textwidth]{img/trace-pc1-pc2-without-r5.png}
\captionof{figure}{Trace PC1 a PC2}
\end{answerfigure}
\end{answer}

\begin{exercise}
Interrumpe las capturas. Analiza el tráfico capturado junto a los
mensajes de depuración de R1 y R4. ¿Envían algún mensaje estos
encaminadores en el momento en que el \emph{tiempo} de algunas entradas
llega a 3 minutos? ¿Por qué? ¿Qué mensajes de las capturas explican que
estos encaminadores descubran las nuevas rutas? ¿Utiliza la
implementación RIP de IOS triggered update?
\end{exercise}

\begin{answer}
Sí, envía las rutas perdidas con métrica 16 a los otros routers, como se
puede ver en la captura inferior de Wireshark. Esto sirve para que el
resto de routers eliminen las rutas a las redes \texttt{40.0.0.0} y
\texttt{41.0.0.0} via R1 de sus tablas más rápidamente, y es lo que se
conoce como \emph{Poison Reverse}.

No, los routers no tienen configurado los triggered updates, sino que se
envían periódicamente.
\end{answer}

\begin{exercise}
Vuelve a configurar de nuevo \emph{RIP} en R5. Observa cómo cambian las
tablas de encaminamiento en R1 y R4. ¿Cuánto tiempo han tardado en
aprender la nueva ruta? ¿Por qué? Comprueba de nuevo cuál es ahora la
ruta que están siguiendo los mensajes intercambiados entre PC1 y PC2 con
\emph{trace}. Incluye aquí la salida.
\end{exercise}

\begin{answer}
Las tablas de encaminamiento se propagan muy rápidamente, como ya hemos
visto anteriormente, dada la RESPONSE inicial de R5 nada más
configurado.

\begin{answerfigure}
\includegraphics[width=\textwidth]{img/trace-pc1-pc2.png}
\captionof{figure}{Trace PC1 a PC2}
\end{answerfigure}
\end{answer}

\begin{exercise}
¿Qué podemos concluir respecto a los tiempos de aprendizaje y
eliminación de rutas?
\end{exercise}

\begin{answer}
Podemos concluir que los tiempos de aprendizaje son mucho más cortos,
porque el router envía un RESPONSE prácticamente inmediatamente.

Sin embargo, los tiempos de eliminación suceden por timeout, por lo que
dependerá de la configuración cuánto tarden en eliminarse.
\end{answer}

\end{document}
