\documentclass{article}

\usepackage{fontspec}
\usepackage{graphicx}
\usepackage{hyperref}
\usepackage{cleveref}
\usepackage{titlesec}
\usepackage{titling}
\usepackage{minted}
\usepackage{natbib}
\usepackage{xcolor}
\usepackage{subcaption}
\usepackage[a4paper, top=1in, bottom=1in]{geometry}

\usepackage[shorthands=off,main=spanish]{babel}

\renewcommand{\listingscaption}{Programa}
\def\listingautorefname~#1\null{(Programa #1)\null}
%\def\figureautorefname~#1\null{(Figura #1)\null}
\def\figureautorefname~#1\null{(#1)\null}

% \newcommand{\sectionbreak}{\clearpage}

\pretitle{\begin{center} \includegraphics[width=200pt]{usal.png} \\ \vspace{3em} \LARGE}
\title{Práctica 2 con Arduino \\ \large Periféricos}
\author{Emilio Cobos Álvarez y Christian Bonal Martín}
\date{11 de Abril de 2016}

\begin{document}

\maketitle

\clearpage
\tableofcontents
\setlength{\parskip}{1em}
\setlength{\parindent}{0pt}
\newpage

\section{Sensor de flexión}

La segunda prueba que realizamos en esta sesión consistía en el montaje y
prueba de un sensor de flexión.

El sensor funciona de forma muy parecida al sensor de temperatura usado en la
sesión anterior, variando la resistencia entre los pines del sensor dependiendo
de la flexión a la que esté sometido.

El ejercicio consistía en lo siguiente:

\begin{itemize}
  \item{Escribir en el monitor serie el voltaje resultante del divisor de
    tensión y encender secuencialmente dos LEDs cuando el sensor vaya estando
    significativamente flexionado.}
\end{itemize}

\subsection{Montaje del circuito}

Lo primero que hicimos fue mirar el
\href{https://cdn.sparkfun.com/datasheets/Sensors/ForceFlex/FLEX\%20SENSOR\%20DATA\%20SHEET\%202014.pdf}{datasheet}
del sensor, y ver la salida que nos daba, sin ningún tipo de LED como salida.

El sensor funciona de forma muy parecida al sensor de temperatura usado en la
sesión anterior, variando la resistencia entre los pines del sensor dependiendo
de la flexión a la que esté sometido.

Posteriormente procedimos a añadir ambos leds de salida.

\begin{figure}[H]
  \includegraphics[width=\linewidth]{../img/flex-basic-circuit.jpg}
  \caption{Circuito flexor básico}
  \label{fig:flex-basic-circuit}
  \centering
  \begin{subfigure}{.4\linewidth}
    \includegraphics[width=.9\linewidth]{../img/flex-middle.png}
    \caption{Salida en reposo}
    \label{fig:flex-middle-out}
  \end{subfigure}
  \begin{subfigure}{.4\linewidth}
    \includegraphics[width=.9\linewidth]{../img/flex-left.png}
    \caption{Salida doblado hacia la izda.}
    \label{fig:flex-left-out}
  \end{subfigure}
  \begin{subfigure}{\linewidth}
    \includegraphics[width=.9\linewidth]{../img/flex-right.png}
    \caption{Salida doblado hacia la dcha.}
    \label{fig:flex-right-out}
  \end{subfigure}
\end{figure}

\subsection{Programación de la placa}

\subsubsection{Sin LEDs}

La versión sin leds del programa es muy sencilla, símplemente hay que convertir
el valor que te da Arduino entre 0 y 1023 a voltios (o milivoltios en este
caso).

\begin{listing}[H]
\inputminted[frame=single, framesep=.5em, linenos, rulecolor=\color{gray}]{java}{../flex-sensor/flex-sensor-1.ino}
\caption{Código del programa con salida via puerto serie}
\end{listing}

\subsubsection{Con LEDs}

La versión con leds del programa lo que hace es encender uno de los LEDs de
salida u otro dependiendo de en qué dirección esté flexionado.

Basándonos en los datos de la parte sin LEDs, programar algo en base a ello fue
muy sencillo.

\begin{listing}[H]
\inputminted[frame=single, framesep=.5em, linenos, rulecolor=\color{gray}]{java}{../flex-sensor/flex-sensor.ino}
\caption{Código del programa con las salidas via LED}
\end{listing}

\subsection{Funcionamiento del circuito}

En el ejercicio realizado, cuanto más se flexiona el sensor dejando las
pastillas en el interior de la curva, más grande es la resistencia que genera el
sensor.

En cambio, si se flexiona de forma que las pastillas de metal estén en el
exterior de la curva, la resistencia de sensor de flexión será menor.

Para controlar el encendido de los LEDs, al realizar la flexión, Arduino nos da
un voltaje (que experimentalmente estaba 300mV y 1500 mV), que transformaremos
para tener un valor entre 0 y 1, para así poder encender un LED u otro en
función del lado hacia el que esté flexionado.

\begin{figure}
  \centering
  \begin{subfigure}{.4\linewidth}
    \includegraphics[width=\linewidth]{../img/flex-circuit-left.jpg}
    \caption{Circuito final con leds (izda.)}
    \label{fig:flex-led-circuit-left}
  \end{subfigure}
  \begin{subfigure}{.4\linewidth}
    \includegraphics[width=\linewidth]{../img/flex-circuit-right.jpg}
    \caption{Circuito final con leds (dcha.)}
    \label{fig:flex-led-circuit-right}
  \end{subfigure}
\end{figure}

\section{Sensor de distancia mediante ultrasonidos HC‐SR04}

La otra prueba que realizamos en esta sesión consistía en el montaje y prueba de
un sensor de distancia mediante ultrasonidos HC-SR04. Estaba compuesto de tres
ejercicios:

\begin{itemize}
  \item{Controlar la señal de salida que envía el sensor hacia un objeto y la
    señal que se recupera tras el reflejo en el objeto a través del pin de
    entrada. Se trata de mostrar en el puerto serie la distancia a la que está
    situado el objeto en cm.}
  \item{Adaptar el programa para controlar eventos, por ejemplo, el paso de un
    vehículo a través de un carril. Se puede comprobar que si la distancia de
    referencia es demasiado grande se pueden perder eventos, ya que la señal
    tarda un tiempo determinado entre emisión y recepción.}
  \item{Encender secuencialmente dos LEDs a medida que el objeto se vaya
    aproximando al sensor.}
\end{itemize}

\subsection{Montaje del circuito}

Primero realizamos el montaje del sensor de ultrasonidos tal y como se nos había
indicado en el guión, con el fin de no crear un daño irreparable al sensor. Los
circuitos que realizamos para los ejercicios con este sensor fueron los
siguientes:

\begin{figure}
  \includegraphics[width=\linewidth]{../img/sound-basic-circuit.jpg}
  \caption{Circuito de detección de distancia sin leds}
  \label{fig:sound-circuit-basic}
\end{figure}

Se adjunta un vídeo del circuito final con los leds en funcionamiento.

\subsection{Programación de la placa}

Hay dos versiones del circuito, una para cada cada uno de las dos
funcionalidades diferentes.

El primero detecta las distancias y enciende la luz dependiendo de si es mayor o
menor de unos límites, y el segundo detecta eventos en base a la primera
lectura.

\begin{listing}[H]
\inputminted[frame=single, framesep=.5em, linenos, rulecolor=\color{gray}]{java}{../sound/sound.ino}
\caption{Código del programa que enciende los LEDs progresivamente}
\end{listing}

\begin{listing}[H]
\inputminted[frame=single, framesep=.5em, linenos, rulecolor=\color{gray}]{java}{../sound/sound-detector-reference.ino}
\caption{Código del programa que detecta eventos}
\end{listing}

\subsection{Funcionamiento del circuito}

En cuanto al ejercicio a, el sensor de ultrasonidos ofrece una medición de $2cm$
a $400cm$, con una precisión que puede alcanzar los $3mm$. El principio básico
de funcionamiento es:

\begin{enumerate}
  \item{Se usa el trigger IO para enviar un pico en una señal (esas son las
    escrituras \texttt{LOW}, \texttt{HIGH}, \texttt{LOW}).

    Nótese que en el ejemplo que vimos del uso de la placa
    (\cite{web:arduino-sound-example})se forzaba un periodo (vía
    \texttt{sleepMicroseconds}) para que fuera detectable, pero resultó no ser
    necesario.}

  \item{El módulo envía automáticamente ocho ciclos de ráfagas de ultrasonidos a
    40 KHz y detecta si existe una señal de vuelta.}

  \item{Esperamos a la señal de vuelta, con la función \texttt{duration}, y el
    tiempo que hemos estado esperando nos sirve para calcular la distancia.}
\end{enumerate}

La fórmula para calcular la distancia ($d$) dada la duración ($t$) es la
siguiente:

$$d = \frac{t}{2} \cdot \frac{10}{291}$$

Una vez tenemos la distancia, podemos hacer lo que queramos con ella.

\section{Referencias comentadas}

La única referencia que hemos utilizado, aparte de la documentación que ya está
comentada en el anterior informe es el ejemplo del detector de distancia
(\cite{web:arduino-sound-example}).

No hay demasiado que comentar, nos salvó un buen rato de mirar el datasheet para
deducir los valores que teníamos que utilizar.

\section{Otros adjuntos}

Aparte del código como es habitual, y todo el contenido del informe, se adjuntan
dos vídeos del funcionamiento del circuito de detección de distancia, en la
carpeta \texttt{video}, uno mostrando el funcionamiento del detector de paso de
objetos, y otro mostrando el segundo programa.

\bibliographystyle{plainnat}
\bibliography{bibliography}

\end{document}
