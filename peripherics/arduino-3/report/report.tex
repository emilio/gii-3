\documentclass{article}

\usepackage{fontspec}
\usepackage{graphicx}
\usepackage{hyperref}
\usepackage{cleveref}
\usepackage{titlesec}
\usepackage{titling}
\usepackage{minted}
\usepackage{natbib}
\usepackage{xcolor}
\usepackage{subcaption}
\usepackage[a4paper, top=1in, bottom=1in]{geometry}

\usepackage[shorthands=off,main=spanish]{babel}

\renewcommand{\listingscaption}{Programa}
\def\listingautorefname~#1\null{(Programa #1)\null}
%\def\figureautorefname~#1\null{(Figura #1)\null}
\def\figureautorefname~#1\null{(#1)\null}

% \newcommand{\sectionbreak}{\clearpage}

\pretitle{\begin{center} \includegraphics[width=200pt]{usal.png} \\ \vspace{3em} \LARGE}
\title{Práctica 3 con Arduino \\ \large Periféricos}
\author{Emilio Cobos Álvarez, Christian Bonal Martín, Alejandro Bodego}
\date{18 de Abril de 2016}

\begin{document}

\maketitle

\clearpage
\tableofcontents
\setlength{\parskip}{1em}
\setlength{\parindent}{0pt}
\newpage

\section{Servomotor}

En esta práctica de actuadores no pudimos realizar pruebas con el motor DC, pero
lo que sí pudimos hacer es usar el servomotor.

El servomotor es relativamente sencillo, y no digamos ya de utilizar gracias a
las bibliotecas de Arduino.

El ejercicio consistía en lo siguiente:

\begin{itemize}
  \item{Se controlará el funcionamiento de un servo como elemento de salida de
      la siguiente manera: se simulará la apertura de una puerta (giro entre 0 y
      90grados) de tal forma que la puerta se va abriendo a medida que el LDR
      detecta más oscuridad.}
\end{itemize}

\subsection{Montaje del circuito}

Lo primero que hicimos fue montar el circuito esquematizado en el guión de la
práctica para poder experimentar y calibrar el servomotor.

Una vez lo tuvimos montado y calibrado, comprobamos (inicialmente
experimentalmente, y luego mirando
\href{https://www.sparkfun.com/products/9065}{la página del
producto}\footnote{Sí, ya sabemos que el orden debería de haber sido el
inverso, pero...}) que sólo soportaba rotación hasta $160\deg$.

Eso nos llevó a hacer un programa de prueba de tal manera que lo único que hacía
era rotar de $0$ a $160\deg$ y viceversa.

El circuito resultante de nuestro primer intento es el montaje básico:

\begin{figure}[H]
  \includegraphics[width=\linewidth]{../img/servo-basic.jpg}
  \caption{Circuito servo básico}
  \label{fig:servo-basic}
\end{figure}

Posteriormente, ya con el LDR integrado (y con las resistencias apropiadas,
basándonos en las primeras prácticas), el resultado es el siguiente:

\begin{figure}[H]
  \includegraphics[width=\linewidth]{../img/servo-ldr.jpg}
  \caption{Circuito servo con LDR}
  \label{fig:servo-ldr}
\end{figure}

\subsection{Programación de la placa}

\subsubsection{Circuito básico}

La versión básica del programa nos sirvió para comprobar cómo funcionaba el
servo, y no va mucho más allá de probar cómo funcionaba la biblioteca que
proporcionaba Arduino:

\begin{listing}[H]
\inputminted[frame=single, framesep=.5em, linenos,
rulecolor=\color{gray}]{cpp}{../servo/servo-roundtrip.ino}
\caption{Programa básico}
\end{listing}

\subsubsection{Col el LDR}

La versión con el LDR tiene algo más de complejidad, ya que no tomamos sólo los
valores máximos y mínimos que obteníamos en esa habitación, sino que se adapta
con el tiempo, cogiendo un valor en la primera lectura, y hayando el máximo y el
mínimo a partir de ahí, pasando de $0$ a $160\deg$ entre ellos.

\begin{listing}[H]
\inputminted[frame=single, framesep=.5em, linenos,
rulecolor=\color{gray}]{cpp}{../servo/servo.ino}
\caption{Código del programa con el LDR}
\end{listing}

\subsection{Funcionamiento del circuito}

Hemos usado el Servo como una caja negra para ser sinceros, pero el
funcionamiento de un servo es (por lo general) bastante directo, simplemente
varía la rotación en base al ancho de la señal de entrada (\textit{Pulse Width
Modulation}), usando un potenciómetro, un motor DC, y un circuito integrado.

Para ver más visualmente cómo funcionaba, ha sido de mucha ayuda
(\cite{web:how-servos-work}).

\section{Pantalla LCD}

En la segunda parte de la práctica teníamos que estudiar el funcionamiento de
los motores DC, pero como la placa de Arduino no nos ofrecía el voltaje
necesario para hacer funcionar el motor, realizamos un circuito básico para
comprobar el funcionamiento de la pantalla LCD de cara a la práctica final.


\begin{figure}[H]
  \includegraphics[width=\linewidth]{../img/lcd.pdf}
  \caption{Pantalla LCD}
  \label{fig:lcd}
\end{figure}

\subsection{Montaje de la placa}

Una vez que obtuvimos el datasheet de nuestro modelo, realizamos las conexiones
según nos indica la imagen superior, usando las salidas digitales de nuestra
placa Arduino. Cabe destacar que para poder visualizar algo en la pantalla
tuvimos que colocar un potenciómetro y usarlo para regular el contraste.

No caímos en sacar una foto al montaje de la placa.

\subsection{Programación de la placa}

Una vez montado, realizamos un programa básico con el fin de que imprimiesemos
por pantalla "Hello World!", aunque el resultado fue bastante decepcionante, ya
que sólo conseguimos ver cuadrados negros (eso tras pasar un buen rato sin ver
nada porque nuestro potenciómetro no estaba correctamente regulado).

\begin{listing}[H]
\inputminted[frame=single, framesep=.5em, linenos,
rulecolor=\color{gray}]{cpp}{../vis/vis.ino}
\caption{"Hello World!" fallido}
\end{listing}

\section{Referencias comentadas}

La única referencia que hemos utilizado para este informe, aparte de las ya
referenciadas datasheets, es el tutorial de arduino para el servomotor
(\cite{web:arduino-servo}).

No hay tampoco demasiado que comentar, fue muy útil para ponerlo a funcionar
rápido.

\section{Otros adjuntos}

Aparte del código como es habitual, y todo el contenido del informe, se adjunta
un vídeo del funcionamiento del servomotor, en la carpeta \texttt{video}.

\bibliographystyle{plainnat}
\bibliography{bibliography}


\end{document}
